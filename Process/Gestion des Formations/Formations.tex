\section{Effectuer une session de formation}
\paragraph{} La figure \ref{effSession} montre le processus "Effectuer une session de formation" . La session doit d'abord être organisée (ce processus est décrit en \ref{sec:Orga}), et sera évaluée, au niveau de l'assimilation individuelle pour chaque participant, et au niveau qualitatif sur la session en elle-même. Ce processus est décrit en section \ref{sec:eval}.

\begin{figure}[H]
	\centering
% \begin{sideways}
	\includegraphics[scale=0.4]{Diagrammes/EffectuerSession.pdf}
% \end{sideways}
	\caption{Session de formation} 
	\label{effSession}
\end{figure}
 	



\section{Organiser une session de formation}
\label{sec:Orga}

\paragraph{}La figure \ref{orgaForm} montre le processus de préparation d'une session de formation. La décision d'organiser une session de formation est prise par la direction des opérations, qui contacte le formateur. Celui-ci pourra ensuite préparer les supports de formation et d'évaluation, et les faire valider par la DO.
\paragraph{} En parallèle, la Direction des Opérations définit une date pour la session, en consultation avec le formateur. Une fois la date arrêtée, la DO peut communiquer sur la session. Commence également le processus de gestion des inscriptions, décrit plus en détail dans la section \ref{sec:inscriptions}. Si suffisamment de participants sont inscrits le jour de la formation, la session peut avoir lieu : la salle peut être réservée et les provisions achetées. Sinon, la direction des opérations doit planifier une nouvelle date et organiser de nouvelles inscriptions.
\paragraph{} En dernier lieu, la DO s'occupe d'imprimer les supports de présentation et la liste des participants.
\begin{figure}[H]
	\centering
\begin{sideways}
	\includegraphics[scale=0.32]{Diagrammes/OrganisationSession.pdf}
\end{sideways}
	\caption{Organisation de session de formation} 
	\label{orgaForm}
\end{figure}




\section{Gérer les inscriptions aux formations}
\label{sec:inscriptions}
\paragraph{} La figure \ref{inscriptionForm} présente le processus de gestion des inscriptions aux formations. Lorsque les inscriptions sont ouvertes, chaque consultant qui s'inscrit est automatiquement ajouté à la liste des participants, jusqu'à ce que le nombre de participants soit atteint. Toute inscription survenant après cela sera ajoutée à la liste d'attente, et pourra participer à la formation si des places se libèrent avant que la formation n'ait lieu.
\paragraph{} Un inscrit peut également se désinscrire, ou se retirer de la liste d'attente. Il devra alors répéter l'opération d'inscription s'il souhaite se réinscrire.
\paragraph{} La veille du jour de la formation, la DO envoie un mail de rappel aux employés présents sur la liste des inscrits. 


\begin{figure}[H]
\centering
\begin{sideways}
	\includegraphics[scale=0.5]{Diagrammes/InscriptionsFormation.pdf}
\end{sideways}
	\caption{Inscription à une session de formation}
	\label{inscriptionForm}
\end{figure}


\section{Évaluer une formation}
\label{sec:eval}
\paragraph{} La figure \ref{evalFormation} présente le processus d'évaluation d'une formation. Après une formation, la DO communique aux participants un questionnaire à remplir pour évaluer la qualité de la formation reçue.
\paragraph{} Les aspects évalués sont les suivants :
\begin{itemize}
	\item Adéquation entre la formation et la communication préalable
	\item Adéquation entre la formation et les objectifs fixés
	\item Adaptation de la durée de la formation par rapport au contenu
	\item Adaptation des supports par rapport au contenu
	\item Qualité de la prestation du formateur
	\item Qualité de la planification
	\item Qualité des fournitures
\end{itemize}

\paragraph{} En dernier lieu sont recueillis les souhaits de formations futures. Les réponses aux questionnaires sont prises en compte dans un but d'amélioration continue.


\begin{figure}[H]
\centering
	\begin{sideways}
		\includegraphics[scale=0.4]{Diagrammes/EvaluerUneFormation.pdf}
	\end{sideways}
	\caption{Évaluation d'une session de formation}
	\label{evalFormation}
\end{figure}


\section{Gérer les souhaits de formation}
\paragraph{} La figure \ref{souhaits} présente le processus de gestion des souhaits de formation. Les souhaits de formation sont recueillis au moment de l'évaluation des formations. Sur souhait de la DO, demande récurrente de consultants, ou bien proposition d'un consultant, la DO recherche un-e formateur-rice sur le sujet si nécessaire, et ajoute la formation au catalogue pour répondre à la demande. 

\begin{figure}[H]
\centering
	% \begin{sideways}
		\includegraphics[scale=0.45]{Diagrammes/GestionDesSouhaits.pdf}
	% \end{sideways}
	\caption{Gestion des souhaits de formation}
	\label{souhaits}
\end{figure}
