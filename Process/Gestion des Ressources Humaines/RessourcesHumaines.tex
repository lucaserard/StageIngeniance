\section{Organiser les visites médicales obligatoires}

\paragraph{} La figure \ref{visitesMedicales} montre le processus de gestion des visites médicales obligatoires.

\paragraph{} Une visite médicale est organisée dans deux cas : lorsque l'employé vient d'arriver, ou bien si au moins deux ans se sont écoulés depuis sa dernière visite médicale. Le responsable administratif contacte le médecin, prend rendez vous pour l'employé et l'en informe. L'employé répond favorablement ou non à la proposition de créneau, et effectue la visite si l'horaire lui convient. Il envoie ensuite le certificat de visite au responsable administratif.
\paragraph{} Si un certain temps s'écoule sans que le certificat soit envoyé, le responsable administratif relance l'employé.

\begin{figure}
	\centering
	\begin{sideways}
	\includegraphics[scale=0.5]{Diagrammes/VisiteMedicale.pdf}
	\end{sideways}
	\caption{Gestion des visites médicales}
	\label{visitesMedicales}	
\end{figure}

\section{Gérer les accès CE}

\paragraph{} La figure \ref{accesCE} montre le processus de gestion des accès CE.

\paragraph{} À l'arrivée de l'employé dans l'entreprise, le RA crée un compte avec des informations de connexion standards, les communique à l'employé qui devra les modifier à sa première connexion.
\paragraph{} L'employé peut ensuite utiliser le site dédié jusqu'à son départ de l'entreprise, au moment duquel le RA supprime son compte.


\begin{figure}
	% \centering
	% \begin{sideways}
	\includegraphics[scale=0.49]{Diagrammes/AccesCE.pdf}
	% \end{sideways}
	\caption{Gestion des accès CE}
	\label{accesCE}	
\end{figure}

\section{Gérer les données de rémunération}

\paragraph{}La figure \ref{MAJDonneesRH} montre le process de gestion des données RH de rémunération. Lorsqu'un Manager veut modifier le salaire d'un de ses N-1, au sortir d'un bilan annuel par exemple, il peut en faire la demande, qui doit être validée par la Direction.



\begin{figure}
	\centering
	% \begin{sideways}
	\includegraphics[scale=0.5]{Diagrammes/MAJDonneesRem.pdf}
	% \end{sideways}
	\caption{Mise à jour des données de rémunération}
	\label{MAJDonneesRH}
\end{figure}




\section{Gérer l'arrivée d'un nouveau collaborateur}

\paragraph{} La figure \ref{easyCome} montre le processus de gestion de l'arrivée d'un nouveau collaborateur.
\paragraph{} Le responsable administratif lui remet tout d'abord un guide d'arrivée. Il lui demande ensuite ses documents administratifs (carte identité/carte séjour, copie diplôme, carte sécu, etc), puis crée et lui communique ses accès CE, et organise la visite médicale.
\paragraph{} Dans le même temps, la Direction des Opération crée un compte ERP pour le collaborateur, et importe ses données candidat de l'ATS. Si le collaborateur est un consultant, la DO lui transmet les instructions pour qu'il renseigne ses compétences dans la cartographie de l'entreprise.\\
Dans le cas d'un interne, la DO fait une demande de création d'un compte sur le domaine Ingéniance, crée un compte sur l'ATS s'il s'agit d'un recruteur, ainsi qu'un compte sur l'application de test si il s'agit d'un recruteur, commercial ou bien opérationnel. 

\begin{figure}
	\centering
	\begin{sideways}
	\includegraphics[scale=0.3]{Diagrammes/NouvelArrivant.pdf}
	\end{sideways}
	\caption{Gestion de l'arrivée d'un collaborateur}
	\label{easyCome}	
\end{figure}



\section{Gérer le départ d'un collaborateur}

\paragraph{} La figure \ref{easyGo}  montre le processus de gestion du départ d'un collaborateur.
\paragraph{} Le responsable administratif met à jour les données RH de l'employé en renseignant sa date de départ, désactivant par là même son accès ERP. Il révoque ensuite son accès CE et son inscription à la mutuelle Ingéniance. Si le collaborateur est un consultant, le RA met également à jour les données de l'affaire sur laquelle il était affecté., ainsi que sa fiche mission. 
\paragraph{} S'il s'agit d'un interne, la DO supprime dans le même temps son compte sur le domaine, désactive celui sur l'ATS s'il s'agit d'un recruteur ainsi que celui l'application de test si il s'agit d'un recruteur, commercial ou bien opérationnel. 


\begin{figure}
	\centering
	\begin{sideways}
	\includegraphics[scale=0.45]{Diagrammes/DepartCollaborateur.pdf}
	\end{sideways}
	\caption{Gestion du départ d'un collaborateur}
	\label{easyGo}	
\end{figure}




\section{Valider les commentaires RH}

\paragraph{} La figure \ref{fichesRH} montre le processus de contrôle des remarques sur les fiches RH. Chaque mois, le N+2 vérifie les fiches remplies par ses N-1, afin de vérifier qu'elles ne contiennent pas de remarques contraires à l'éthique et à l'esprit de l'entreprise.


\begin{figure}
	\centering
	% \begin{sideways}
	\includegraphics[scale=0.6]{Diagrammes/ControleFichesRH.pdf}
	% \end{sideways}
	\caption{Validation des fiches RH}
	\label{fichesRH}	
\end{figure}