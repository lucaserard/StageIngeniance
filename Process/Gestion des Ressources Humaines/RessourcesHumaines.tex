\section{Organiser les visites médicales obligatoires}

\paragraph{} La figure \ref{visitesMedicales} montre le processus de gestion des visites médicales obligatoires.

\paragraph{} Une visite médicale est organisée dans deux cas : lorsque l'employé vient d'arriver, ou bien si au moins deux ans se sont écoulés depuis sa dernière visite médicale. Le responsable administratif contacte le médecin, prend rendez vous pour l'employé et l'en informe. L'employé répond favorablement ou non à la proposition de créneau, et effectue la visite si l'horaire lui convient. Il envoie ensuite le certificat de visite au responsable administratif.
\paragraph{} Si un certain temps s'écoule sans que le certificat soit envoyé, le responsable administratif relance l'employé.

\begin{figure}
	\centering
	\begin{sideways}
	\includegraphics[scale=0.5]{Diagrammes/VisiteMedicale.pdf}
	\end{sideways}
	\caption{Gestion des visites médicales}
	\label{visitesMedicales}	
\end{figure}

\section{Gérer les accès CE}

\paragraph{} La figure \ref{accesCE} montre le processus de gestion des accès CE.

\paragraph{} À l'arrivée de l'employé dans l'entreprise, le RA crée un compte avec des informations de connexion standards, les communique à l'employé qui devra les modifier à sa première connexion.
\paragraph{} L'employé peut ensuite utiliser le site dédié jusqu'à son départ de l'entreprise, au moment duquel le RA supprime son compte.


\begin{figure}
	% \centering
	% \begin{sideways}
	\includegraphics[scale=0.49]{Diagrammes/AccesCE.pdf}
	% \end{sideways}
	\caption{Gestion des accès CE}
	\label{accesCE}	
\end{figure}

\section{Gérer les données employé}

\paragraph{}La figure \ref{MAJDonneesRH} montre le process de gestion des données RH. Lorsque le RA reçoit d'un employé une notification de changement de situation, il met à jour les données RH pour refléter ce changement.



\begin{figure}
	\centering
	% \begin{sideways}
	\includegraphics[scale=0.7]{Diagrammes/MAJDonneesRH.pdf}
	% \end{sideways}
	\caption{Gestion des données employé}
	\label{MAJDonneesRH}	
\end{figure}