\section{Gérer le référentiel de compétences}

\paragraph{} La figure \ref{gestionReferentiel} montre le processus de maintien du référentiel de compétences. Lorsque le service recrutement ou la Direction des opération désire modifier le référentiel, les deux entités doivent se réunir pour valider ensemble les changements. La DO et la direction du Recrutement mettent ensuite 
respectivement à jour les données de l'ERP et de l'ATS.

\begin{figure}
	\centering
	% \begin{sideways}
	\includegraphics[scale=0.4]{Diagrammes/GererReferentiel.pdf}
	% \end{sideways}
	\caption{Gestion du référentiel de compétences}
	\label{gestionReferentiel}
\end{figure}


\section{Gérer ses compétences individuelles}


\paragraph{} La figure \ref{gestionCompIndiv} montre le processus de gestion des compétences individuelles. Le consultant à toute liberté de modifier ses compétences. Le Directeur des Opérations est cependant notifié lorsqu'il le fait.

\begin{figure}
	\centering
	% \begin{sideways}
	\includegraphics[scale=0.5]{Diagrammes/GererPropresCompetences.pdf}
	% \end{sideways}
	\caption{Gestion des compétences individuelles}
	\label{gestionCompIndiv}
\end{figure}


\section{Monitorer les compétences au niveau entreprise}

\paragraph{} La figure \ref{monitoringComp} montre le processus de monitoring des compétences au niveau entreprise. Chaque mois, le Directeur des Opérations extrait les données de compétences, génère les visuels associés, et peut prendre des actions en conséquence.


\begin{figure}
	\centering
	% \begin{sideways}
	\includegraphics[scale=0.5]{Diagrammes/Monitoring.pdf}
	% \end{sideways}
	\caption{Monitoring des compétences}
	\label{monitoringComp}
\end{figure}