\section{Processus de recrutement}

\paragraph{} La figure \ref{recrutementFull} montre le processus de recrutement dans son ensemble.
\paragraph{} Le point de départ est un besoin client relayé par l'équipe commercial. Ce besoin peut être précis et ponctuel, mais il s'agit plus généralement d'un ensemble de profils génériques, basés sur des compétences recherchées dans les BU.
\paragraph{} L'équipe recrutement effectue ensuite un sourcing de candidats en se basant sur ces besoins, puis leur fait passer une suite d'entretiens. Le workflow d'entretien est détaillé dans la partie \ref{sec:entretiens}.
\paragraph{} Une fois qu'un candidat est considéré comme apte à intégrer la structure, la direction lui fait une proposition.

\begin{figure}
	\begin{sideways}
	\centering
	\includegraphics[scale=0.3]{Diagrammes/RecrutementComplet.pdf}
	\end{sideways}
	\caption{Processus de recrutement}
	\label{recrutementFull}	
\end{figure}

\section{Workflow d'entretiens}
\label{sec:entretiens}

\paragraph{} La figure \ref{entretiens} présente le workflow des entretiens.
\paragraph{} La première étape après avoir repéré un candidat est de le contacter et d'effectuer un premier contact téléphonique, afin de mieux cerner les expériences et compétences du candidat potentiel, ainsi que ses disponibilités et attentes en terme de poste. À l'issu de ce premier contact, le candidat peut être déclaré inapte à travailler chez Ingéniance, auquel cas il son dossier est définitivement refusé. Il peut également être renvoyé dans le vivier candidat, par exemple si son profil est interessant mais qu'il n'est pas disponible pour le moment. 
\paragraph{} Si le recruteur et le candidat décident de continuer, un premier entretien est organisé. 
\paragraph{} Si l'échange est concluant, le candidat sera présenté à l'équipe commerciale qui donnera son avis sur le candidat, et peut décider de le rencontrer ou non. L'issue de l'entretien déterminera si le candidat sera engagé ou non. 
\paragraph{}À chacune des étapes, le dossier du candidat peut également passer au statut "rejeté" ou "vivier".


\begin{figure}
	\begin{sideways}
	\centering
	\includegraphics[scale=0.3]{Diagrammes/EntretiensWF.pdf}
	\end{sideways}
	\caption{Workflow d'entretiens}
	\label{entretiens}	
\end{figure}