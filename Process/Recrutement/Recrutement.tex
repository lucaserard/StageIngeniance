\section{Processus de recrutement}

\paragraph{} La figure \ref{recrutementFull} montre le processus de recrutement dans son ensemble.
\paragraph{} Le point de départ est un besoin client relayé par l'équipe commerciale. Ce besoin peut être précis et ponctuel, mais il s'agit plus généralement d'un ensemble de profils génériques, basés sur des compétences recherchées dans les BU.
\paragraph{} L'équipe recrutement effectue ensuite un sourcing de candidats en se basant sur ces besoins, puis leur fait passer une suite d'entretiens. Le workflow d'entretien est détaillé dans la partie \ref{sec:entretiens}. Le candidat peut se qualifier pour la suite du processus, être refusé, ou bien retourner en vivier "Projet".
\paragraph{} Dans le cadre d'un recrutement sur profil, la direction fait une proposition au candidat, et il démarre en InterContrat s'il accepte.
\paragraph{} S'il s'agit d'un recrutement sur Projet, le candidat doit d'abord passer par une qualification avec le client. Il ne reçoit de proposition que si l'équipe commerciale obtient un GO client. S'il accepte la proposition, il démarre directement sa mission chez le client.

\begin{figure}
	\centering
	\begin{sideways}
	\includegraphics[scale=0.3]{Diagrammes/RecrutementComplet.pdf}
	\end{sideways}
	\caption{Processus de recrutement}
	\label{recrutementFull}	
\end{figure}

\section{Workflow d'entretiens}
\label{sec:entretiens}

\paragraph{} La figure \ref{entretiens} présente le workflow des entretiens.
\paragraph{} La première étape après avoir repéré un candidat est de le contacter et d'effectuer un premier contact téléphonique, afin de mieux cerner les expériences et compétences du candidat potentiel, ainsi que ses disponibilités et attentes en terme de poste. À l'issu de ce premier contact, le candidat peut être déclaré inapte à travailler chez Ingéniance, auquel cas il son dossier est définitivement refusé. Il peut également être renvoyé dans le vivier candidat, par exemple si son profil est interessant mais qu'il n'est pas disponible pour le moment. 
\paragraph{} Si le recruteur et le candidat décident de continuer, un premier entretien est organisé. 
\paragraph{} Si l'échange est concluant, le candidat sera présenté à l'équipe commerciale qui donnera son avis sur le candidat, et peut décider de le rencontrer ou non. L'issue de l'entretien déterminera si le candidat sera engagé ou non. 
\paragraph{}À chacune des étapes, le dossier du candidat peut également passer au statut "rejeté" ou "vivier".


\begin{figure}
	\centering
	\begin{sideways}
	\includegraphics[scale=0.3]{Diagrammes/EntretiensWF.pdf}
	\end{sideways}
	\caption{Workflow d'entretiens}
	\label{entretiens}	
\end{figure}

\section{Gérer ses candidats}

\paragraph{} La figure \ref{gestionCandidats} montre le processus de gestion des candidats par les recruteurs dans le cadre de leur portefeuille.
\paragraph{} Lorsqu'un CV entre dans le scope d'Ingéniance, tous les recruteurs contactent le candidat. Le premier réussissant à obtenir un contact direct l'ajoute à son portefeuille candidat. Il le rencontre ensuite si le candidat est disponible et interessé. Sinon, le recruteur récupère ses informations de disponibilité et peut le recontacter plus tard.

\begin{figure}
	\centering
	% \begin{sideways}
	\includegraphics[scale=0.5]{Diagrammes/GererUnCandidat.pdf}
	% \end{sideways}
	\caption{Gestion des candidats}
	\label{gestionCandidats}	
\end{figure}

\section{Sourcer un candidat}

\paragraph{} La figure \ref{sourcing} montre le processus de sourcing de candidats. Les recruteurs cherchent des profils sur les JobBoards et les réseaux sociaux, puis les incorporent dans la base de candidat. Si des profils sont déja présents dans la base, ils seront mis à jour avec les informations les plus récentes. Si ce sont de nouveaux profils, ils seront ajoutés au vivier "candidats à contacter".

\begin{figure}
	\centering
	\begin{sideways}
	\includegraphics[scale=0.5]{Diagrammes/SourcerUnCandidat.pdf}
	\end{sideways}
	\caption{Sourcing de candidats}
	\label{sourcing}	
\end{figure}

