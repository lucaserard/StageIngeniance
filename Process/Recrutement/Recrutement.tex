 \paragraph{} Dans cette partie sont évoqués les viviers candidats. Il en existe différentes sortes, et le type de vivier dans lequel est envoyé un candidat dépend de nombreux facteurs qui ne seront pas traités ici.

 \section{Processus de recrutement}
\label{section:recrutFull}
\paragraph{} La figure \ref{recrutementFull} montre le processus de recrutement dans son ensemble.
\paragraph{} Le point de départ est un besoin client relayé par l'équipe commerciale. Ce besoin peut être précis et ponctuel, mais il s'agit plus généralement d'un ensemble de profils génériques, basés sur des compétences recherchées dans les BU.
\paragraph{} L'équipe recrutement effectue ensuite un sourcing (processus détaillé en partie \ref{section:sourcing}) de candidats en se basant sur ces besoins, les répartit entre recruteurs (processus détaillé dans la partie \ref{section:portefeuille} ) puis leur fait passer une suite d'entretiens. Le workflow d'entretien est détaillé dans la partie \ref{sec:entretiens}. Le candidat peut se qualifier pour la suite du processus ou être renvoyé en vivier.
\paragraph{} Dans le cadre d'un recrutement sur profil, la direction fait une proposition au candidat, et il est engagé s'il accepte.
\begin{figure}
	\centering
	\begin{sideways}
	\includegraphics[scale=0.3]{Diagrammes/RecrutementComplet.pdf}
	\end{sideways}
	\caption{Processus de recrutement}
	\label{recrutementFull}	
\end{figure}

\section{Workflow d'entretiens}
\label{sec:entretiens}

\paragraph{} La figure \ref{entretiens} présente le workflow des entretiens.
 
\paragraph{} Si le premier entretien est concluant, le candidat sera présenté à l'équipe commerciale qui donnera son avis sur le candidat, et peut décider de le rencontrer ou non. L'issue de l'entretien déterminera si le candidat recevra une proposition, sur profil ou sur projet, ou s'il sera renvoyé en vivier.

\begin{figure}
	\centering
	% \begin{sideways}
	\includegraphics[scale=0.4]{Diagrammes/EntretiensWF.pdf}
	% \end{sideways}
	\caption{Workflow d'entretiens}
	\label{entretiens}	
\end{figure}

\section{Contacter et attribuer les candidats}
\label{section:portefeuille}
\paragraph{} La figure \ref{gestionCandidats} montre le processus de contact et de répartition des candidats entre les recruteurs.
\paragraph{} Lorsqu'un CV entre dans le scope d'Ingéniance, tous les recruteurs contactent le candidat. Le premier réussissant à obtenir un contact direct l'ajoute à son portefeuille candidat. Il le rencontre ensuite si le candidat est exploitable dans l'immédiat. 
\begin{figure}
	\centering
	% \begin{sideways}
	\includegraphics[scale=0.6]{Diagrammes/GererUnCandidat.pdf}
	% \end{sideways}
	\caption{Gestion des candidats}
	\label{gestionCandidats}	
\end{figure}

\section{Sourcer des candidats}
\label{section:sourcing}
\paragraph{} La figure \ref{sourcing} montre le processus de sourcing de candidats. Les recruteurs cherchent des profils sur les JobBoards et les réseaux sociaux, puis les incorporent dans la base de candidats, en même temps que les profils provenant de candidatures diverses (cooptation, réponse à annonce, candidature spontanée). Si des profils sont déjà présents dans la base, ils seront mis à jour avec les informations les plus récentes. Si ce sont de nouveaux profils, ils seront ajoutés au vivier "candidats à contacter".

\begin{figure}
	\centering
	% \begin{sideways}
	\includegraphics[scale=0.48]{Diagrammes/SourcerUnCandidat.pdf}
	% \end{sideways}
	\caption{Sourcing de candidats}
	\label{sourcing}	
\end{figure}

\section{Publier des offres}

\paragraph{} La figure \ref{publierOffres} montre le processus de publication d'offres d'emploi. Un recruteur rédige l'annonce, qui doit être validée par le directeur du recrutement. Une fois validée, elle est diffusée sur diverses plate-formes d'emploi et sites d'école.


\begin{figure}
	\centering
	% \begin{sideways}
	\includegraphics[scale=0.6]{Diagrammes/PublierOffres.pdf}
	% \end{sideways}
	\caption{Publication d'offres}
	\label{publierOffres}	
\end{figure}



