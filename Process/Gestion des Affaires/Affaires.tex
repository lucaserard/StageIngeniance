\section{Assistance Technique Unitaire (ATU)}

\subsection{Mise en place d'une affaire ATU}

\paragraph{}La figure \ref{mepATU} présente le processus de mise en place d'une affaire ATU. Le processus débute avec l'appel d'offre d'un client, auquel un membre de l'équipe commerciale va répondre en proposant un ou plusieurs consultants.\\
Si le client considère le profil convient à ses besoins, il rencontrera le consultant au cours d'un ou plusieurs entretiens, à l'issue desquels il donnera ou non son accord pour démarrer la mission du consultant. S'en suit une phase de contractualisation, visible sur la figure \ref{contractATU}
\paragraph{} Le contrat est rédigé par le client, puis mis à disposition des équipes commerciales sur une plate-forme d'achat dédiée. Le contrat est téléchargé, imprimé, signé, puis transmis au Responsable Administratif pour envoi postal au client.\\
Si nécessaire, le consultant signe des accords supplémentaires, notamment les règles déontologiques et de confidentialité, etc.

\begin{figure}[H]
% \begin{sideways}
	\centering
	\includegraphics[scale=0.3]{Diagrammes/MiseEnPlaceATU.pdf}
% \end{sideways}
	\caption{Mise en place d'une ATU} 
	\label{mepATU}
\end{figure}
	
	
\begin{figure}[H]
% \begin{sideways}
	\centering
	\includegraphics[scale=0.35]{Diagrammes/ContractualisationATU.pdf}
% \end{sideways}
	\caption{Contractualisation d'une ATU} 
	\label{contractATU}
\end{figure}

\subsection{Clôturer une affaire ATU}

\paragraph{} Une affaire ATU peut se clôturer dans trois cas de figure : le consultant arrive au terme de sa mission, le client décide de mettre fin au contrat, ou bien le consultant démissionne.
\paragraph{} Le contrat se conclut donc simplement, avec un arrêt de la facturation et une classification du contrat.

\begin{figure}[H]
% \begin{sideways}
	\centering
	\includegraphics[scale=0.4]{Diagrammes/ClotureATU.pdf}
% \end{sideways}
	\caption{Clôture d'une ATU} 
	\label{contractATU}
\end{figure}


\subsection{Prolonger une affaire ATU}

\paragraph{} La figure \ref{prolongerATU} présente le processus de prolongation d'une affaire ATU. Le responsable administratif reçoit un avenant au contrat de la part du client, le transmet à la direction qui le signe. Le consultant signe les engagements complémentaires, puis le Responsable Administratif renvoie l'avenant et les pièces jointes au client, au format papier ou électronique.

\begin{figure}[H]
% \begin{sideways}
	\centering
	\includegraphics[scale=0.4]{Diagrammes/RenouvelerATU.pdf}
% \end{sideways}
	\caption{Prolongation d'une ATU} 
	\label{prolongerATU}
\end{figure}

\newpage

\section{Assistance Technique Globale}

\subsection{Mise en place d'une affaire ATG}

 \paragraph{} La figure \ref{miseEnPlaceATG} montre le processus de mise en place d'une ATG. Lorsqu'au moins 5 consultants sont déjà placés chez un même client en ATU, le client fait une demande de massification de contrats, et envoie un cahier des charges auquel l'équipe commerciale devra répondre pour permettre la mise en place du contrat. La demande peut également originer de l'équipe commerciale.
 \paragraph{} Pour ce faire, l'équipe commerciale rédige un PAQ présentant les capacités de l'entreprise à répondre aux exigences du client, en s'appuyant sur une batterie de KPI. Si le client estime les éléments présentés suffisants, l'affaire peut être contractualisée. Ce processus est décrit plus en détail dans la partie \ref{subsection:contratATG}.

\begin{figure}[H]
% \begin{sideways}
	\centering
	\includegraphics[scale=0.4]{Diagrammes/MiseEnPlaceATG.pdf}
% \end{sideways}
	\caption{Mise en place d'une ATG} 
	\label{miseEnPlaceATG}
\end{figure}


\subsection{Contractualisation d'une ATG}
\label{subsection:contratATG}

\paragraph{} La figure \ref{contratATG} montre le processus de contractualisation d'une affaire ATG.
\paragraph{} Le contrat est d'abord envoyé par le client à l'équipe commerciale, puis signé par la Direction, et ensuite renvoyé au client par le Responsable Administratif. 


\begin{figure}[H]
% \begin{sideways}
	\centering
	\includegraphics[scale=0.4]{Diagrammes/ContractualisationATG.pdf}
% \end{sideways}
	\caption{Contractualisation d'une ATG} 
	\label{contratATG}
\end{figure}



\subsection{Assignation de ressources à une ATG}

\paragraph{} Dans le cadre d'une ATG, un contact informel avec le client permet de lancer une recherche de ressource, sans appel d'offre ni contractualisation.
\paragraph{}Après contact avec le client, l'équipe commerciale propose un ou plusieurs consultants pouvant correspondre à son besoin. Comme pour une ATU, le client valide les profils au travers d'entretiens, et donne ou non son feu vert. Une Fiche de Gestion du Changement est alors émise par le client, et téléchargée par l'équipe commerciale depuis la plate-forme d'achat dédiée. Le consultant peut ensuite commencer sa mission.
Ce processus est présenté dans la figure \ref{asgnRcesATG}.

\begin{figure}[H]
% \begin{sideways}
	\centering
	\includegraphics[scale=0.3]{Diagrammes/AssignationRessourcesATG.pdf}
% \end{sideways}
	\caption{Assignation de ressources en ATG} 
	\label{asgnRcesATG}
\end{figure}


\subsection{Retrait d'une ressource en ATG}
\paragraph{} la figure \ref{retraitATG} montre le processus de retrait d'une ressource d'une affaire ATG. Cela peut survenir lorsque la mission du consultant s'achève, sur démission du consultant ou sur interruption de la mission par le client. Une FGC est émise, et le consultant arrête la mission.


\begin{figure}[H]
% \begin{sideways}
	\centering
	\includegraphics[scale=0.5]{Diagrammes/RetraitRessourceATG.pdf}
% \end{sideways}
	\caption{Retrait de ressources en ATG} 
	\label{retraitATG}
\end{figure}
