\section{Assistance Technique Unitaire (ATU)}

\subsection{Mise en place d'une affaire ATU}

\paragraph{}La figure \ref{mepATU} présente le processus de mise en place d'une affaire ATU. Le processus débute avec l'appel d'offre d'un client, auquel un membre de l'équipe commerciale va répondre en proposant un ou plusieurs consultants.\\
Si le client considère le profil convient à ses besoins, il rencontrera le consultant au cours d'un ou plusieurs entretiens, à l'issue desquels il donnera ou non son accord pour démarrer la mission du consultant. S'en suit une phase de contractualisation, visible sur la figure \ref{contractATU}
\paragraph{} Le contrat est rédigé par le client, puis mis à disposition des équipes commerciales sur une plateforme d'achat dédiée. Le contrat est téléchargé, imprimé, signé, puis transmis au Responsable Administratif pour envoi postal au client.\\
Si nécessaire, le consultant signe des accords supplémentaires, notamment les règles déontologiques et de confidentialité, etc.

\begin{figure}[H]
% \begin{sideways}
	\centering
	\includegraphics[scale=0.3]{Diagrammes/MiseEnPlaceATU.pdf}
% \end{sideways}
	\caption{Mise en place d'une ATU} 
	\label{mepATU}
\end{figure}
	
	
\begin{figure}[H]
% \begin{sideways}
	\centering
	\includegraphics[scale=0.35]{Diagrammes/ContractualisationATU.pdf}
% \end{sideways}
	\caption{Contractualisation d'une ATU} 
	\label{contractATU}
\end{figure}

\subsection{Clôturer une affaire ATU}

\paragraph{} Une affaire ATU peut se clôturer dans trois cas de figure : le consultant arrive au terme de sa mission, le client décide de mettre fin au contrat, ou bien 
le consultant démissionne.
\paragraph{} Le contrat se conclut donc simplement, avec un arrêt de la facturation et une classification du contrat.

\begin{figure}[H]
% \begin{sideways}
	\centering
	\includegraphics[scale=0.4]{Diagrammes/ClotureATU.pdf}
% \end{sideways}
	\caption{Contractualisation d'une ATU} 
	\label{contractATU}
\end{figure}




\section{Assistance Technique Globale}

\subsection{Mise en place d'une affaire ATG}

\subsection{Assignement de ressources à une ATG}

\paragraph{} Dans le cadre d'une ATG, un contact informel avec le client permet de lancer une recherche de ressource, sans appel d'offre ni contractualisation.\\
\paragraph{}Après contact avec le client, l'équipe commerciale propose un ou plusieurs consultants pouvant correspondre à son besoin. Comme pour une ATU, le client valide les profils au travers d'entretiens, et donne ou non son feu vert. Une Fiche de Gestion du Changement est alors émise par le client, et téléchargée par l'équipe commerciale depuis la plate-forme d'achat dédiée. Le consultant peut ensuite commencer sa mission.\\
Ce processus est présenté dans la figure \ref{asgtRcesATG}



\begin{figure}[H]
% \begin{sideways}
	\centering
	\includegraphics[scale=0.3]{Diagrammes/AssignementRessourcesATG.pdf}
% \end{sideways}
	\caption{\textcolor{red}{Assignation de ressources en ATG}} 
	\label{asgtRcesATG}
\end{figure}


