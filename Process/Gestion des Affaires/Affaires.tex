\section{Assistance Technique Unitaire (ATU)}

\subsection{Mise en place d'une affaire ATU}

\paragraph{}La figure \ref{mepATU} présente le processus de mise en place d'une affaire ATU. Le processus débute avec l'appel d'offre d'un client, auquel un membre de l'équipe commerciale va répondre en proposant un ou plusieurs consultants.\\ Pour cela, il doit d'abord sélectionner un candidat, parmi les consultants en IC ou bien parmi les candidats au recrutement en attente de projet (cf \ref{section:recrutFull}).
Si le client considère que le profil convient à ses besoins, il rencontrera le consultant au cours d'un ou plusieurs entretiens, à l'issue desquels il donnera ou non son accord pour démarrer la mission du consultant. À la réception d'un GO client, le BM peut créer les données de l'affaire. Toute l'équipe commerciale est avertie que le consultant concerné n'est plus disponible.\\ De même, si un message de ce type arrive durant le processus avant obtention d'un GO client, le commercial doit chercher un autre candidat à présenter.
Une fois la confirmation client obtenue, la phase de contractualisation, détaillée dans la partie \ref{subsection:contractATU}, peut démarrer.

\paragraph{} Le responsable Administratif peut ensuite compléter les données de l'affaire, et initialiser celles de production.

\begin{figure}[H]
% \begin{sideways}
	\centering
	\includegraphics[scale=0.3]{Diagrammes/MiseEnPlaceATU.pdf}
% \end{sideways}
	\caption{Mise en place d'une ATU} 
	\label{mepATU}
\end{figure}
	
	
\subsection{Contractualisation d'une ATU}
\label{subsection:contractATU}

\paragraph{} La figure \ref{contractATU} montre le processus de contractualisation d'une affaire ATU.
\paragraph{} Le contrat est rédigé par le client, puis mis à disposition des équipes commerciales sur une plate-forme d'achat dédiée, ou par e-mail. Le contrat est téléchargé, imprimé, signé par la direction, puis transmis au Responsable Administratif pour envoi postal ou électronique au client.\\
Si nécessaire, le consultant signe des accords supplémentaires, notamment les règles déontologiques et de confidentialité, etc.
\begin{figure}[H]
% \begin{sideways}
	\centering
	\includegraphics[scale=0.35]{Diagrammes/ContractualisationATU.pdf}
% \end{sideways}
	\caption{Contractualisation d'une ATU} 
	\label{contractATU}
\end{figure}

\subsection{Clôturer une affaire ATU}

\paragraph{} Une affaire ATU peut se clôturer dans trois cas de figure : le consultant arrive au terme de sa mission, le client décide de mettre fin au contrat, ou bien le consultant démissionne.
\paragraph{} Le contrat se conclut donc simplement, avec un arrêt de la facturation et la clôture des données de production associées.

\begin{figure}[H]
% \begin{sideways}
	\centering
	\includegraphics[scale=0.4]{Diagrammes/ClotureATU.pdf}
% \end{sideways}
	\caption{Clôture d'une ATU} 
	\label{clotureATU}
\end{figure}


\subsection{Prolonger une affaire ATU}

\paragraph{} La figure \ref{prolongerATU} présente le processus de prolongation d'une affaire ATU. Le BM initie la procédure en mettant à jour les informations commerciales qui auraient changé avec la prolongation de l'affaire. Une fois que le responsable administratif reçoit l'avenant au contrat, il complète les informations contractuelles et transmet à la direction pour signature. Le consultant signe les engagements complémentaires, puis le Responsable Administratif renvoie l'avenant et les pièces jointes au client, au format papier ou électronique.

\begin{figure}[H]
% \begin{sideways}
	\centering
	\includegraphics[scale=0.4]{Diagrammes/RenouvelerATU.pdf}
% \end{sideways}
	\caption{Prolongation d'une ATU} 
	\label{prolongerATU}
\end{figure}

\newpage

\section{Assistance Technique Globale}

\subsection{Mise en place d'une affaire ATG}

 \paragraph{} La figure \ref{miseEnPlaceATG} montre le processus de mise en place d'une ATG. Lorsqu'au moins 5 consultants sont déjà placés chez un même client en ATU, le client fait une demande de massification de contrats, et envoie un cahier des charges auquel l'équipe commerciale devra répondre pour permettre la mise en place du contrat. La demande peut également originer de l'équipe commerciale.
 \paragraph{} Pour ce faire, la Direction des Opérations rédige un Dossier Technique présentant la capacité (compétences, expertises, références) de l'entreprise à répondre aux exigences du client, ainsi qu'un Plan d'Assurance Qualité. Si le client estime les éléments présentés suffisants, l'affaire peut être contractualisée. Sinon, le client peut demander des éléments supplémentaires, ou refuser la proposition. Ce processus est décrit plus en détail dans la partie \ref{subsection:contratATG}.

\begin{figure}[H]
% \begin{sideways}
	\centering
	\includegraphics[scale=0.32]{Diagrammes/MiseEnPlaceATG.pdf}
% \end{sideways}
	\caption{Mise en place d'une ATG} 
	\label{miseEnPlaceATG}
\end{figure}


\subsection{Contractualisation d'une ATG}
\label{subsection:contratATG}

\paragraph{} La figure \ref{contratATG} montre le processus de contractualisation d'une affaire ATG.
\paragraph{} Le contrat est d'abord envoyé par le client à l'équipe commerciale, puis signé par la Direction, et ensuite renvoyé au client par le Responsable Administratif. 


\begin{figure}[H]
% \begin{sideways}
	\centering
	\includegraphics[scale=0.4]{Diagrammes/ContractualisationATG.pdf}
% \end{sideways}
	\caption{Contractualisation d'une ATG} 
	\label{contratATG}
\end{figure}



\subsection{Assignation de ressources à une ATG}

\paragraph{} Dans le cadre d'une ATG, un contact informel avec le client permet de lancer une recherche de ressource, sans appel d'offre ni contractualisation.
\paragraph{}Après contact avec le client, l'équipe commerciale recherche et propose un ou plusieurs consultants pouvant correspondre à son besoin. Comme pour une ATU, le client valide les profils au travers d'entretiens, et donne ou non son feu vert. Une Fiche de Gestion du Changement est alors émise par le client, et téléchargée par l'équipe commerciale depuis la plate-forme d'achat dédiée. Le consultant peut ensuite commencer sa mission, et le responsable administratif initialiser les données de production associées, en affectant le consultant à l'affaire.
Ce processus est présenté dans la figure \ref{asgnRcesATG}.

\begin{figure}[H]
	\centering
\begin{sideways}
	\includegraphics[scale=0.35]{Diagrammes/AssignationRessourcesATG.pdf}
\end{sideways}
	\caption{Assignation de ressources en ATG} 
	\label{asgnRcesATG}
\end{figure}


\subsection{Retrait d'une ressource en ATG}
\paragraph{} La figure \ref{retraitATG} montre le processus de retrait d'une ressource d'une affaire ATG. Cela peut survenir lorsque la mission du consultant s'achève, sur démission du consultant ou sur interruption de la mission par le client. Une FGC est émise, le consultant arrête la mission, et le Responsable Administratif met à jour les données de production associées. 


\begin{figure}[H]
% \begin{sideways}
	\centering
	\includegraphics[scale=0.5]{Diagrammes/RetraitRessourceATG.pdf}
% \end{sideways}
	\caption{Retrait de ressources en ATG} 
	\label{retraitATG}
\end{figure}




