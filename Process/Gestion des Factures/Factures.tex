\section{Émission et validation de factures}

\paragraph{} La figure \ref{validationFactures} présente le workflow de validation des factures. Les factures sont générées par le Responsable Administratif, qui en contrôle la conformité pour les valider. La direction est ensuite notifiée, effectue un contrôle à son tour, et les valide. Le responsable administratif est notifié de la validation et peut envoyer les factures aux clients.


	
\begin{figure}
	\centering
	\begin{sideways}
		\includegraphics[scale=0.5]{Diagrammes/FacturesWF.pdf}
	\end{sideways}
	\caption{Émission de factures}
	\label{validationFactures}	
\end{figure}

\section{Factures et régulation ATG}
\paragraph{} La figure \ref{facturesATG} présente le processus d'émission et de régulation des factures ATG. Chaque mois, le reponsable Administratif reçoit du client les prévisions pour le mois suivant. Des factures sont générées et complétées à partir de ces prévisions. Ce sous processus est détaillé sur la figure \ref{previsionnelATG}.
\paragraph{} Au bout d'un temps variable selon les clients, le responsable administratif émet des factures de régulation à partir des temps effectifs des consultants.   

\begin{figure}
	\centering
	% \begin{sideways}
		\includegraphics[scale=0.4]{Diagrammes/FacturesATG.pdf}
	% \end{sideways}
	\caption{Gestion des factures et régulations ATG}
	\label{facturesATG}	
\end{figure}	


\begin{figure}
	\centering
	% \begin{sideways}
		\includegraphics[scale=0.4]{Diagrammes/FacturesPrevisionnelles.pdf}
	% \end{sideways}
	\caption{Émission de factures prévisionnelles}
	\label{validationFactures}	
\end{figure}


\section{Gestion des litiges}
\paragraph{À Formaliser}


\section{Refacturation des notes de frais}
\paragraph{} la figure \ref{NDFRefacturables} montre le processus de refacturation des notes de frais. Lorsqu'une note de frais refacturable est validée, le Responsable Administratif génère une facture à partir de cette NDF. S'ensuit une validation administrative, puis une validation de la Direction. En dernier lieu, la facture est envoyée au client.


\begin{figure}
	\centering
	% \begin{sideways}
		\includegraphics[scale=0.4]{Diagrammes/NDFRefacturables.pdf}
	% \end{sideways}
	\caption{Refacturation de Notes de Frais}
	\label{NDFRefacturables}	
\end{figure}


\section{Monitoring des DSO et CA client}

\paragraph{} La figure \ref{monitoringCO} présente le processus de monitoring des \textit{cuts off} client (DSO et CA). Le responsable financier extrait les données depuis ses outils métiers, les visualise, et peut ensuite en tenir compte pour les mois à venir.


\begin{figure}
	\centering
	% \begin{sideways}
		\includegraphics[scale=0.4]{Diagrammes/Monitoring.pdf}
	% \end{sideways}
	\caption{Refacturation de Notes de Frais}
	\label{NDFRefacturables}	
\end{figure}
