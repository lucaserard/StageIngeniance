\section{Gestion des droits des applications}

\paragraph{} Modifier les droits dans une application peut s'effectuer via deux scénarios : \begin{itemize}
	\item Un employé change de groupe
	\item Les droits d'un groupe évoluent
\end{itemize}

\paragraph{} Dans les deux cas, le changement de droit est soumis à une validation effectuée en amont, que ce soit dans le cadre d'un changement de service pour le premier cas, ou de la modification d'un processus pour le second.
\paragraph{} La figure \ref{gestionDroits} montre ce processus.


\begin{figure}
	\centering
	% \begin{sideways}
	\includegraphics[scale=0.6]{Diagrammes/GererDroits.pdf}
	% \end{sideways}
	\caption{Gestion des droits sur les applications}
	\label{gestionDroits}
\end{figure}



\section{Gérer l'évolution des application internes}



\paragraph{} la figure \ref{evolutionsSoftInternes} montre le processus de gestion des évolutions dans le cas d'une application ayant fait l'objet d'un développement en interne.
\paragraph{} Lorsque qu'un utilisateur fait une demande d'évolution, la Direction des Opérations l'examine et si la demande est jugée réalisable et pertinente, elle sera implémentée dans l'application. Il faut d'abord identifier un ou plusieurs consultants suffisamment qualifiés pour assurer ce développement. On entre ensuite dans un cycle complet de développement, de la formalisation du besoin à la recette, pour finir par le déploiement en production de la fonctionnalité demandée.
\begin{figure}
	\centering
	\begin{sideways}
	\includegraphics[scale=0.5]{Diagrammes/GererEvolutionsSoftInterne.pdf}
	\end{sideways}
	\caption{Gestion des évolutions sur les applications internes}
	\label{evolutionsSoftInternes}
\end{figure}




\section{Gérer l'évolution des application Externes}

\paragraph{} la figure \ref{evolutionsSoftExternes} montre le processus de gestion des évolutions dans le cas d'une application ayant fait l'objet d'une acquisition auprès d'un éditeur.

\paragraph{} Lorsque qu'un utilisateur fait une demande d'évolution, la Direction des Opérations l'examine et si la demande est jugée réalisable et pertinente, elle sera implémentée dans l'application. La demande est présentée à l'éditeur, avec qui le développement est planifié. On entre ensuite dans un cycle complet de développement, de la formalisation du besoin à la recette, pour finir par le déploiement en production de la fonctionnalité demandée.

\begin{figure}
	\centering
	\begin{sideways}
	\includegraphics[scale=0.45]{Diagrammes/GererEvolutionsSoftExterne.pdf}
	\end{sideways}
	\caption{Gestion des évolutions sur les applications externes}
	\label{evolutionsSoftExternes}
\end{figure}



