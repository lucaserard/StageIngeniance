\documentclass[12pt]{scrreprt}
\usepackage[T1]{fontenc}
\usepackage[utf8]{inputenc}
\usepackage{lmodern}
\usepackage{textcomp}
\usepackage[francais]{babel, varioref}
\usepackage{graphicx}
\usepackage{listings}
\usepackage{xspace}
\usepackage{amsmath}
\usepackage{amssymb}
\usepackage{calc}
\usepackage{listingsutf8}
\usepackage{color}
\usepackage{xcolor}
% \usepackage{afterpage}
% \usepackage[style=verbose-note,backend=bibtex]{biblatex}
\usepackage{url}
\usepackage[top=2.1cm,bottom=2.2cm,left=2cm,right=2cm]{geometry}
\usepackage[final]{pdfpages}
\usepackage{pslatex}
\usepackage{rotating}
\usepackage{longtable}
\usepackage{colortbl}
\usepackage{eurosym}
\usepackage{$Ingeniance/Ressources/templateIngenianceCentralise}
\usepackage{hyperref} % Créer des liens et des signets
\hypersetup{
colorlinks=true, %colorise les liens
breaklinks=true, %permet le retour à la ligne dans les liens trop longs
urlcolor= blue, %couleur des hyperliens
linkcolor= black, %couleur des liens internes
citecolor=black,  %couleur des références
}



%Ajouter le nom du dossier contenant le texte et les diagrammes liés à un nouveau macro process 
\graphicspath{
			{"Gestion des Affaires"/}
			{"Gestion des Temps"/}
			{"Gestion de la Remuneration"/}
			{"Gestion des Notes de Frais"/}
            {Recrutement/}
            {"Gestion des Formations"/}
            {"Gestion des Factures"/}
			}


\renewcommand\titre{Processus Ingéniance}
\renewcommand\soustitre{Document récapitulatif}
\author{Direction des opérations}
\date{Avril 2016}
\begin{document}

\titleIngeniance{0}{}{0}{150}{220}{}

%table des matières
\tableofcontents

\initIngeniance

%--------------------------------------------------------------------------
%intro
\chapter{Introduction}

\section{Présentation du document}
\paragraph{} Le présent document a pour objectif de lister et d'expliciter les processus en œuvre dans la société Ingéniance.
\paragraph{} Les processus sont représentés via la norme BPMN2.0 : l’enchaînement des différentes tâches est figuré via des boites reliées entre elles par des flèches. La modélisation implique également différents événements, par exemple des messages, ou des événements liés à un \textit{timer}.


\section{Terminologie et abréviations}
\paragraph{} Le tableau \ref{abbr} liste et explicite tous les abréviations et acronymes utilisés dans le document.
\begin{table}[H]
\begin{center}
    \begin{longtable}{|p{0.2\linewidth}|p{0.75\linewidth}|}
    \hline  % Une ligne horizontale
    \rowcolor[gray]{.8}
    Abréviation & Signification\\
    \hline
    ATG & Assistance Technique Unitaire\\
    \hline
    ATU & Assistance Technique Groupée\\
    \hline
    BM & Business Manager\\
    \hline
    BU & Business Unit \\
    \hline
    CP & Congés Payés\\
    \hline
    CRA & Compte Rendu d'Activité\\
    \hline
    DO & Direction des Opérations\\
    \hline
    KPI & Key Performance Indicators\\
    \hline
    PAQ & Plan d'Assurance Qualité\\
    \hline
    \end{longtable}
    \caption{Terminologie}
    \label{abbr}
\end{center}
\end{table}



%------------------------------------------------------
% ----------Process----------------------

\chapter{Gestion des temps}

\section{Demande d'absence}
On peut voir sur la figure \ref{conges} le workflow de demande d'absences. Une demande est issue d'un employé, consultant ou interne, et doit être validée, d'abord par son N+1 puis par le responsable administratif. À chaque étape du processus, l'employé est averti en cas de refus et peut modifier sa demande pour la soumettre à nouveau. Une notification est également envoyée à l'employé à la validation de sa demande.
\begin{figure}
	\centering
	\begin{sideways}
		\includegraphics[scale=0.5]{Diagrammes/DemandeCongeWF.pdf}
	\end{sideways}
	\caption{Workflow de demande d'absence}
	\label{conges}
\end{figure}

\section{Saisie des temps}

On peut voir sur la figure \ref{tps} le workflow de saisie des temps. En fin de mois, chaque consultant doit soumettre un rapport d'activité, avec les pièces jointes nécessaires, à son Manager dans lequel il indique le détail de ses journées de travail et ses absences du mois. Ce comptre rendu d'activité, ou CRA, doit être validé par le Business Manager, puis par le responsable administratif. À chaque étape du processus, l'employé est averti en cas de refus et doit modifier et soumettre à nouveau son CRA jusqu'à validation.
\begin{figure}
	\centering
	\begin{sideways}
		\includegraphics[scale=0.5]{Diagrammes/SaisieDesTempsWF.pdf}
	\end{sideways}
	\caption{Workflow de saisie des temps}
	\label{tps}
\end{figure}


\section{Gestion des relances}
\paragraph{} On peut voir sur la figure \ref{relances} le processus de relance des consultants. Lorsque la fin du mois approche, chaque consultant reçoit un mail lui rappelant de soumettre son Compte Rendu d'activité pour le mois écoulé. À la suite de ce mail, une relance quotidienne est envoyée jusqu'à ce que le consultant ait soumis un CRA correctement complété.
\begin{figure}
\centering
	\includegraphics[scale=0.5]{Diagrammes/Relances.pdf}
	\caption{Processus de relance des consultants sur les CRA}
	\label{relances}
\end{figure}


\section{Gestion des arrêts maladie}
\paragraph{} La figure \ref{maladie} le processus de gestion des arrêts maladie. Un employé malade doit contacter son manager, faire une demande d'arrêt maladie, puis envoyer son certificat médical.
\paragraph{} Toutes les 24h jusqu'à recevoir le certificat médical, le Responsable Administratif relance l'employé. Si 72h s'écoulent sans réception du certificat, l'arrêt maladie est déclaré absence injustifiée.

\begin{figure}
\centering
	\includegraphics[scale=0.5]{Diagrammes/CongesMaladie.pdf}
	\caption{Processus de gestion des arrêts maladie}
	\label{maladie}
\end{figure}



\chapter{Gestion de la rémunération}
\section{Demande de prime}
\paragraph{} La figure \ref{primes} montre le workflow de demande de primes. Une demande de prime est effectuée par un Manager, pour un consultant. Dans ce cas, on entend par "Manager" le Directeur des Opérations, un Business Manager, Directeur de BU, ou bien un membre de la Direction.
\paragraph{} Dans le cas d'un Business Manager, sa demande passe d'abord par son Directeur de BU, avant d'être validée par la Direction Générale. Sinon, cela passe directement à la Direction Générale pour validation.



\begin{figure}[H]
% \begin{sideways}
	\centering
	\includegraphics[scale=0.5]{Diagrammes/PrimesWF.pdf}
% \end{sideways}
	\caption{Demande et validation de primes} 
	\label{primes}
\end{figure}

\section{Établissement de la paie}

\paragraph{} La figure \ref{paie} présente le processus d'établissement de la paie. La dernière semaine du mois, le Responsable Financier vérifie les changements éventuels dans les données RH des employés, et met à jour les données de Sage Paie le cas échéant.
\paragraph{} Il peut ensuite importer les données de variables de paie dans Sage Paie. Les données en question comptent : \begin{itemize}
	\item les congés \begin{itemize}
		\item Congés Payés;
		\item Congés Sans Solde;
		\item Arrêt Maladie;
		\item Congé exceptionnel;
		\item RTT;
	\end{itemize}
	\item les paniers repas;
	\item les remboursements de titres de transport;
	\item les primes;
\end{itemize}
\paragraph{} Si des données manquent à l'import (non remise du rapport d'activité mensuel par un consultant), on utilise les données d'absence disponibles grâce aux demandes de congés. Le consultant est averti de l'absence de son RMA, et le responsable Financier prévoit une régulation pour le mois suivant dans le cas où le temps de travail réel du consultant ne correspondrait pas aux données déduites.

\paragraph{} La paie est ensuite calculée à partir de ces variables. Dans le même temps, les compteurs de congés et de RTT sont mis à jour. La paie est finalement virée sur les compte en banque des employés, et le processus se finit avec l'émission des bulletins de paie.

\begin{figure}[H]
\centering
\begin{sideways}
	\includegraphics[scale=0.42]{Diagrammes/Paie.pdf}
\end{sideways}
	\caption{Établissement de la paie} 
	\label{paie}
\end{figure}


\section{Gestion des titres de transports}

\paragraph{} La figure \ref{transports} montre le processus de gestion des titres de transport. Un employé remet son titre de transport au responsable administratif . Ce dernier vérifie le titre, et en saisit la valeur dans le système si il est valide. Dans le cas contraire, le responsable administratif contacte l'employé qui doit le soumettre de nouveau après avoir résolu le problème.

\begin{figure}[H]
\centering
% \begin{sideways}
	\includegraphics[scale=0.5]{Diagrammes/Transport.pdf}
% \end{sideways}
	\caption{Gestion des titres de transport} 
	\label{transports}
\end{figure}

\section{Gestion des paniers repas}
\paragraph{} La figure \ref{repas} montre le processus de gestion des paniers repas. Le responsable administratif décompte les jours entiers travaillés par chaque employé, et en soustrait les jours où l'employé a soumis une note de frais liée à un repas. Le calcul des paniers repas peut ensuite être effectué, sur la base de 5 euros par jour décompté précédemment.



\begin{figure}[H]
\centering
% \begin{sideways}
	\includegraphics[scale=0.5]{Diagrammes/PaniersRepas.pdf}
% \end{sideways}
	\caption{Gestion des paniers repas}
	\label{repas}
\end{figure}







\chapter{Gestion des Notes de Frais}
\section{Gestion des notes de frais internes}

\paragraph{} Lorsqu'un interne fait une demande de remboursement de note de frais, elle doit être validée d'abord par son N+1 avant de passer à la Direction générale qui donnera un accord définitif. Le responsable administratif pourra ensuite effectuer le remboursement. Ce \textit{workflow} est décrit sur la figure \ref{ndfI}.
\paragraph{} Si le N+1 s'avère être la direction, une seule validation est requise.


\begin{figure}[H]
% \begin{sideways}
	\centering
	\includegraphics[scale=0.5]{Diagrammes/NoteDeFraisWFInterne.pdf}
% \end{sideways}
	\caption{Demande et validation de notes de frais internes} 
	\label{ndfI}
\end{figure}

\section{Gestion des notes de frais consultant}

\paragraph{} Lorsqu'un consultant fait une demande de remboursement de note de frais, elle doit être validée par le Responsable Administratif , puis par son Business Manager, avant de passer par la Direction Générale qui donnera un accord définitif. Le Responsable Financier pourra ensuite effectuer le remboursement. Ce \textit{workflow} est décrit sur la figure \ref{ndfC}.

\begin{figure}[H]
% \begin{sideways}
	\centering
	\includegraphics[scale=0.47]{Diagrammes/NoteDeFraisWFConsultant.pdf}
% \end{sideways}
	\caption{Demande et validation de notes de frais consultants}
	\label{ndfC}
\end{figure}


\chapter{Gestion des Affaires}
\section{Assistance Technique Unitaire (ATU)}

\subsection{Mise en place d'une affaire ATU}

\paragraph{}La figure \ref{mepATU} présente le processus de mise en place d'une affaire ATU. Le processus débute avec l'appel d'offre d'un client, auquel un membre de l'équipe commerciale va répondre en proposant un ou plusieurs consultants.\\
Si le client considère le profil convient à ses besoins, il rencontrera le consultant au cours d'un ou plusieurs entretiens, à l'issue desquels il donnera ou non son accord pour démarrer la mission du consultant. S'en suit une phase de contractualisation, visible sur la figure \ref{contractATU}
\paragraph{} Le contrat est rédigé par le client, puis mis à disposition des équipes commerciales sur une plateforme d'achat dédiée. Le contrat est téléchargé, imprimé, signé, puis transmis au Responsable Administratif pour envoi postal au client.\\
Si nécessaire, le consultant signe des accords supplémentaires, notamment les règles déontologiques et de confidentialité, etc.

\begin{figure}[H]
% \begin{sideways}
	\centering
	\includegraphics[scale=0.3]{Diagrammes/MiseEnPlaceATU.pdf}
% \end{sideways}
	\caption{Mise en place d'une ATU} 
	\label{mepATU}
\end{figure}
	
	
\begin{figure}[H]
% \begin{sideways}
	\centering
	\includegraphics[scale=0.35]{Diagrammes/ContractualisationATU.pdf}
% \end{sideways}
	\caption{Contractualisation d'une ATU} 
	\label{contractATU}
\end{figure}

\subsection{Clôturer une affaire ATU}

\paragraph{} Une affaire ATU peut se clôturer dans trois cas de figure : le consultant arrive au terme de sa mission, le client décide de mettre fin au contrat, ou bien 
le consultant démissionne.
\paragraph{} Le contrat se conclut donc simplement, avec un arrêt de la facturation et une classification du contrat.

\begin{figure}[H]
% \begin{sideways}
	\centering
	\includegraphics[scale=0.4]{Diagrammes/ClotureATU.pdf}
% \end{sideways}
	\caption{Contractualisation d'une ATU} 
	\label{contractATU}
\end{figure}




\section{Assistance Technique Globale}

\subsection{Mise en place d'une affaire ATG}

\subsection{Assignement de ressources à une ATG}

\paragraph{} Dans le cadre d'une ATG, un contact informel avec le client permet de lancer une recherche de ressource, sans appel d'offre ni contractualisation.\\
\paragraph{}Après contact avec le client, l'équipe commerciale propose un ou plusieurs consultants pouvant correspondre à son besoin. Comme pour une ATU, le client valide les profils au travers d'entretiens, et donne ou non son feu vert. Une Fiche de Gestion du Changement est alors émise par le client, et téléchargée par l'équipe commerciale depuis la plate-forme d'achat dédiée. Le consultant peut ensuite commencer sa mission.\\
Ce processus est présenté dans la figure \ref{asgtRcesATG}



\begin{figure}[H]
% \begin{sideways}
	\centering
	\includegraphics[scale=0.3]{Diagrammes/AssignementRessourcesATG.pdf}
% \end{sideways}
	\caption{\textcolor{red}{Assignation de ressources en ATG}} 
	\label{asgtRcesATG}
\end{figure}




\chapter{Gestion des Formations}
\section{Effectuer une session de formation}
\paragraph{} La figure \ref{effSession} montre le processus "Effectuer une session de formation" . La session doit d'abord être organisée (ce processus est décrit en \ref{sec:Orga}), et sera évaluée, au niveau de l'assimilation individuelle pour chaque participant, et au niveau qualitatif sur la session en elle-même. Ce processus est décrit en section \ref{sec:eval}.

\begin{figure}[H]
	\centering
\begin{sideways}
	\includegraphics[scale=0.5]{Diagrammes/EffectuerSession.pdf}
\end{sideways}
	\caption{Éxecution d'une session de formation} 
	\label{effSession}
\end{figure}
 	



\section{Organiser une session de formation}
\label{sec:Orga}

\paragraph{}La figure \ref{orgaForm} montre le processus de préparation d'une session de formation. La décision d'organiser une session de formation est prise par la direction des opérations, qui contacte le formateur. Celui-ci pourra ensuite préparer les supports de formation et d'évaluation, et les faire valider par la DO.
\paragraph{} En parallèle, la Direction des Opérations définit une date pour la session, en consultation avec le formateur. Une fois la date arrêtée, la DO peut communiquer sur la session. Commence également le processus de gestion des inscriptions, décrit plus en détail dans la section \ref{sec:inscriptions}. Si suffisamment de participants sont inscrits le jour de la formation, la session peut avoir lieu : la salle peut être réservée et les provisions achetées. Sinon, la direction des opérations doit planifier une nouvelle date et organiser de nouvelles inscriptions.
\paragraph{} En dernier lieu, la DO s'occupe d'imprimer les supports de présentation et la liste des participants.
\begin{figure}[H]
	\centering
\begin{sideways}
	\includegraphics[scale=0.32]{Diagrammes/OrganisationSession.pdf}
\end{sideways}
	\caption{Organisation de session de formation} 
	\label{orgaForm}
\end{figure}




\section{Gérer les inscriptions aux formations}
\label{sec:inscriptions}
\paragraph{} La figure \ref{inscriptionForm} présente le processus de gestion des inscriptions aux formations. Lorsque les inscriptions sont ouvertes, chaque consultant qui s'inscrit est automatiquement ajouté à la liste des participants, jusqu'à ce que le nombre de participants soit atteint. Toute inscription survenant après cela sera ajoutée à la liste d'attente, et pourra participer à la formation si des places se libèrent avant que la formation n'ait lieu.
\paragraph{} Un inscrit peut également se désinscrire, ou se retirer de la liste d'attente. Il devra alors répéter l'opération d'inscription s'il souhaite se réinscrire.
\paragraph{} La veille du jour de la formation, la DO envoie un mail de rappel aux employés présents sur la liste des inscrits. 


\begin{figure}[H]
\centering
\begin{sideways}
	\includegraphics[scale=0.5]{Diagrammes/InscriptionsFormation.pdf}
\end{sideways}
	\caption{Gestion des inscriptions à une session de formation}
	\label{inscriptionForm}
\end{figure}


\section{Évaluer une formation}
\label{sec:eval}
\paragraph{} La figure \ref{evalFormation} présente le processus d'évaluation d'une formation. Après une formation, la DO communique aux participants un questionnaire à remplir pour évaluer la qualité de la formation reçue.
\paragraph{} Les aspects évalués sont les suivants :
\begin{itemize}
	\item Adéquation entre la formation et la communication préalable
	\item Adéquation entre la formation et les objectifs fixés
	\item Adaptation de la durée de la formation par rapport au contenu
	\item Adaptation des supports par rapport au contenu
	\item Qualité de la prestation du formateur
	\item Qualité de la planification
	\item Qualité des fournitures
\end{itemize}

\paragraph{} En dernier lieu sont recueillis les souhaits de formations futures. Les réponses aux questionnaires sont prises en compte dans un but d'amélioration continue.


\begin{figure}[H]
\centering
	\begin{sideways}
		\includegraphics[scale=0.4]{Diagrammes/EvaluerUneFormation.pdf}
	\end{sideways}
	\caption{Évaluation d'une session de formation}
	\label{evalFormation}
\end{figure}


\section{Gérer les souhaits de formation}
\paragraph{} La figure \ref{souhaits} présente le processus de gestion des souhaits de formation. Les souhaits de formation sont recueillis au moment de l'évaluation des formations. Sur souhait de la DO, demande récurrente de consultants, ou bien proposition d'un consultant, la DO recherche un-e formateur-rice sur le sujet si nécessaire, et ajoute la formation au catalogue pour répondre à la demande. 

\begin{figure}[H]
\centering
	% \begin{sideways}
		\includegraphics[scale=0.45]{Diagrammes/GestionDesSouhaits.pdf}
	% \end{sideways}
	\caption{Gestion des souhaits de formation}
	\label{souhaits}
\end{figure}


\section{Ajouter une formation au catalogue}
\paragraph{} La figure \ref{addFormation} présente le processus d'ajout d'une formation au catalogue de formations. La DO contacte le formateur, qui élabore un plan de session et le communique à la DO. Si le plan est validé, la formation est ajoutée au catalogue.


\begin{figure}[H]
\centering
	% \begin{sideways}
		\includegraphics[scale=0.45]{Diagrammes/AjoutFormationCatalogue.pdf}
	% \end{sideways}
	\caption{Ajout d'une formation au catalogue}
	\label{addFormation}
\end{figure}

\chapter{Gestion des Factures}
\section{Émission et validation de factures}

\paragraph{} La figure \ref{validationFactures} présente le workflow de validation des factures. Les factures sont générées par le Responsable Administratif, qui en contrôle la conformité pour les valider. Elles passent ensuite par la Direction pour une nouvelle validation. Elles sont finalement envoyées au client par le responsable administratif, après renseignement de la date d'envoi dans les données de suivi.
	
\begin{figure}
	\centering
	\begin{sideways}
		\includegraphics[scale=0.5]{Diagrammes/FacturesATU.pdf}
	\end{sideways}
	\caption{Émission de factures}
	\label{validationFactures}	
\end{figure}

\section{Factures mensuelles en ATG}
\paragraph{} La figure \ref{facturesATG} présente le processus d'émission et de régulation des factures ATG. Chaque mois, le reponsable Administratif reçoit du client les prévisions pour le mois suivant. Des factures sont générées et complétées à partir de ces prévisions. Ce sous processus est détaillé sur la figure \ref{previsionnelATG}.
\paragraph{} Au bout d'un temps variable selon les clients, une campagne de régularisation est initiée par le client.    

\begin{figure}
	\centering
	% \begin{sideways}
		\includegraphics[scale=0.4]{Diagrammes/FacturesATG.pdf}
	% \end{sideways}
	\caption{Gestion des factures et régulations ATG}
	\label{facturesATG}	
\end{figure}	


\begin{figure}
	\centering
	% \begin{sideways}
		\includegraphics[scale=0.4]{Diagrammes/FacturesPrevisionnelles.pdf}
	% \end{sideways}
	\caption{Émission de factures prévisionnelles}
	\label{previsionnelATG}	
\end{figure}


\section{Régularisation en ATG}

\paragraph{} La figure \ref{regulATG} montre le processus d'établissement des régularisation en ATG. Le client lance une campagne de régularisation et prend contact avec le Responsable Administratif. Une rencontre est organisée, au cours de laquelle sont discutés les écarts entre les prévisions et les temps effectifs ainsi que les litiges sur les temps effectués. La Direction des Opérations effectue ensuite une mise à jour des données de suivi de chaque lot.
\paragraph{} Les donneurs d'ordre côté client génèrent ensuite chacun une FGC de régularisation, qu'ils envoient au RA. Lorsque celui ci les reçoit, il peut générer les factures de régularisation. Ces factures sont ensuites validées et envoyées au client.


\begin{figure}
	\centering
	% \begin{sideways}
		\includegraphics[scale=0.4]{Diagrammes/RegularisationsATG.pdf}
	% \end{sideways}
	\caption{Régularisations en ATG}
	\label{regulATG}	
\end{figure}	



\section{Gestion des litiges}
\paragraph{À Formaliser}


\section{Refacturation des notes de frais}
\paragraph{}La figure \ref{NDFRefacturables} montre le processus de refacturation des notes de frais. Lorsqu'une note de frais refacturable est validée, le Responsable Administratif génère une facture à partir de cette NDF. S'ensuit une validation administrative, puis une validation de la Direction. En dernier lieu, la facture est envoyée au client.


\begin{figure}
	\centering
	% \begin{sideways}
		\includegraphics[scale=0.4]{Diagrammes/NDFRefacturables.pdf}
	% \end{sideways}
	\caption{Refacturation de Notes de Frais}
	\label{NDFRefacturables}	
\end{figure}


\section{Monitoring des DSO et CA client}

\paragraph{} La figure \ref{monitoringCO} présente le processus de monitoring des \textit{cuts off} client (DSO et CA). Le responsable financier extrait les données depuis ses outils métiers, les visualise, et peut ensuite mettre en place des actions si nécessaire.

\begin{figure}
	\centering
	% \begin{sideways}
		\includegraphics[scale=0.4]{Diagrammes/Monitoring.pdf}
	% \end{sideways}
	\caption{Monitoring des cuts off client}
	\label{monitoringCO}	
\end{figure}

\chapter{Recrutement}
\section{Processus de recrutement}

\paragraph{} La figure \ref{recrutementFull} montre le processus de recrutement dans son ensemble.
\paragraph{} Le point de départ est un besoin client relayé par l'équipe commercial. Ce besoin peut être précis et ponctuel, mais il s'agit plus généralement d'un ensemble de profils génériques, basés sur des compétences recherchées dans les BU.
\paragraph{} L'équipe recrutement effectue ensuite un sourcing de candidats en se basant sur ces besoins, puis leur fait passer une suite d'entretiens. Le workflow d'entretien est détaillé dans la partie \ref{sec:entretiens}.
\paragraph{} Une fois qu'un candidat est considéré comme apte à intégrer la structure, la direction lui fait une proposition.

\begin{figure}
	\begin{sideways}
	\centering
	\includegraphics[scale=0.3]{Diagrammes/RecrutementComplet.pdf}
	\end{sideways}
	\caption{Processus de recrutement}
	\label{recrutementFull}	
\end{figure}

\section{Workflow d'entretiens}
\label{sec:entretiens}

\paragraph{} La figure \ref{entretiens} présente le workflow des entretiens.
\paragraph{} La première étape après avoir repéré un candidat est de le contacter et d'effectuer un premier contact téléphonique, afin de mieux cerner les expériences et compétences du candidat potentiel, ainsi que ses disponibilités et attentes en terme de poste. À l'issu de ce premier contact, le candidat peut être déclaré inapte à travailler chez Ingéniance, auquel cas il son dossier est définitivement refusé. Il peut également être renvoyé dans le vivier candidat, par exemple si son profil est interessant mais qu'il n'est pas disponible pour le moment. 
\paragraph{} Si le recruteur et le candidat décident de continuer, un premier entretien est organisé. 
\paragraph{} Si l'échange est concluant, le candidat sera présenté à l'équipe commerciale qui donnera son avis sur le candidat, et peut décider de le rencontrer ou non. L'issue de l'entretien déterminera si le candidat sera engagé ou non. 
\paragraph{}À chacune des étapes, le dossier du candidat peut également passer au statut "rejeté" ou "vivier".


\begin{figure}
	\begin{sideways}
	\centering
	\includegraphics[scale=0.3]{Diagrammes/EntretiensWF.pdf}
	\end{sideways}
	\caption{Workflow d'entretiens}
	\label{entretiens}	
\end{figure}
% \section{Assistance Technique Unitaire (ATU)}

\subsection{Mise en place d'une affaire ATU}

\paragraph{}La figure \ref{mepATU} présente le processus de mise en place d'une affaire ATU. Le processus débute avec l'appel d'offre d'un client, auquel un membre de l'équipe commerciale va répondre en proposant un ou plusieurs consultants.\\
Si le client considère le profil convient à ses besoins, il rencontrera le consultant au cours d'un ou plusieurs entretiens, à l'issue desquels il donnera ou non son accord pour démarrer la mission du consultant. S'en suit une phase de contractualisation, visible sur la figure \ref{contractATU}
\paragraph{} Le contrat est rédigé par le client, puis mis à disposition des équipes commerciales sur une plateforme d'achat dédiée. Le contrat est téléchargé, imprimé, signé, puis transmis au Responsable Administratif pour envoi postal au client.\\
Si nécessaire, le consultant signe des accords supplémentaires, notamment les règles déontologiques et de confidentialité, etc.

\begin{figure}[H]
% \begin{sideways}
	\centering
	\includegraphics[scale=0.3]{Diagrammes/MiseEnPlaceATU.pdf}
% \end{sideways}
	\caption{Mise en place d'une ATU} 
	\label{mepATU}
\end{figure}
	
	
\begin{figure}[H]
% \begin{sideways}
	\centering
	\includegraphics[scale=0.35]{Diagrammes/ContractualisationATU.pdf}
% \end{sideways}
	\caption{Contractualisation d'une ATU} 
	\label{contractATU}
\end{figure}

\subsection{Clôturer une affaire ATU}

\paragraph{} Une affaire ATU peut se clôturer dans trois cas de figure : le consultant arrive au terme de sa mission, le client décide de mettre fin au contrat, ou bien 
le consultant démissionne.
\paragraph{} Le contrat se conclut donc simplement, avec un arrêt de la facturation et une classification du contrat.

\begin{figure}[H]
% \begin{sideways}
	\centering
	\includegraphics[scale=0.4]{Diagrammes/ClotureATU.pdf}
% \end{sideways}
	\caption{Contractualisation d'une ATU} 
	\label{contractATU}
\end{figure}




\section{Assistance Technique Globale}

\subsection{Mise en place d'une affaire ATG}

\subsection{Assignement de ressources à une ATG}

\paragraph{} Dans le cadre d'une ATG, un contact informel avec le client permet de lancer une recherche de ressource, sans appel d'offre ni contractualisation.\\
\paragraph{}Après contact avec le client, l'équipe commerciale propose un ou plusieurs consultants pouvant correspondre à son besoin. Comme pour une ATU, le client valide les profils au travers d'entretiens, et donne ou non son feu vert. Une Fiche de Gestion du Changement est alors émise par le client, et téléchargée par l'équipe commerciale depuis la plate-forme d'achat dédiée. Le consultant peut ensuite commencer sa mission.\\
Ce processus est présenté dans la figure \ref{asgtRcesATG}



\begin{figure}[H]
% \begin{sideways}
	\centering
	\includegraphics[scale=0.3]{Diagrammes/AssignementRessourcesATG.pdf}
% \end{sideways}
	\caption{\textcolor{red}{Assignation de ressources en ATG}} 
	\label{asgtRcesATG}
\end{figure}





\end{document}
