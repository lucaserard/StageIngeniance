\documentclass[12pt]{scrreprt}
\usepackage[T1]{fontenc}
\usepackage[utf8]{inputenc}
\usepackage{lmodern}
\usepackage{textcomp}
\usepackage[francais]{babel, varioref}
\usepackage{graphicx}
\usepackage{listings}
\usepackage{xspace}
\usepackage{amsmath}
\usepackage{amssymb}
\usepackage{calc}
\usepackage{listingsutf8}
\usepackage{color}
\usepackage{xcolor}
\usepackage{afterpage}
% \usepackage[style=verbose-note,backend=bibtex]{biblatex}
\usepackage{url}
\usepackage[top=2.1cm,bottom=2.2cm,left=2cm,right=2cm]{geometry}
\usepackage[final]{pdfpages}
\usepackage{pslatex}
\usepackage{rotating}
\usepackage{longtable}
\usepackage{colortbl}
\usepackage{eurosym}
 




% Pour sommaire cliquable 
\usepackage{hyperref} % Créer des liens et des signets
\hypersetup{
colorlinks=true, %colorise les liens
breaklinks=true, %permet le retour à la ligne dans les liens trop longs
urlcolor= blue, %couleur des hyperliens
linkcolor= black, %couleur des liens internes
citecolor=black,  %couleur des références
}

\usepackage{$Ingeniance/Ressources/templateIngenianceCentralise}

\graphicspath{
			{"Gestion des Affaires"/}
			{"Gestion des Temps"/}
			{"Gestion de la Remuneration"/}
			{"Gestion des Notes de Frais"/}
            {Recrutement/}
            {"Gestion des Formations"/}
			}


\renewcommand\titre{Processus Ingéniance}
\renewcommand\soustitre{Document récapitulatif}
\author{Direction des opérations}
\date{Avril 2016}
\begin{document}

% \makeatletter
% \def\input@path{{$Ingeniance/Process/"Gestion des Affaires}}
% \makeatother


\titleIngeniance{0}{}{0}{150}{220}{}
\tableofcontents
\chapter{Introduction}

\section{Terminologie et abréviations}
\paragraph{} Le tableau \ref{abbr} liste et explicite tous les abréviations et acronymes utilisés dans le document.
\begin{table}[H]
\begin{center}
    \begin{longtable}{|p{0.2\linewidth}|p{0.75\linewidth}|}
    \hline  % Une ligne horizontale
    \rowcolor[gray]{.8}
    Abréviation & Signification\\
    \hline
    BM & Business Manager\\
    \hline
    BU & Business Unit \\
    \hline
    CP & Congés Payés\\
    \hline
    CRA & Compte Rendu d'Activité\\
    \hline
    DO & Direction des Opérations\\
    \hline
    RMA & Rapport Mensuel d'Activité\\
    \hline
    \end{longtable}
    \caption{Terminologie}
    \label{abbr}
\end{center}
\end{table}

\section{Remarques sur les schémas}
\paragraph{} Le lecteur notera qu'à l'intérieur des diagrammes, les différents acteurs sont représentés par des boites, et non par des Pools et Lanes.














\chapter{Gestion des temps}

\section{Demande de congé}
On peut voir sur la figure \ref{conges} le workflow de demande de congés. Une demande est issue d'un employé, consultant ou interne, et doit être validée, d'abord par son N+1 puis par le responsable administratif. À chaque étape du processus, l'employé est averti en cas de refus et peut modifier sa demande pour la soumettre à nouveau.
\begin{figure}
	\includegraphics[scale=0.5]{Diagrammes/DemandeCongeWF.pdf}
	\caption{Workflow de demande de congés}
	\label{conges}
\end{figure}

\section{Saisie des temps}

On peut voir sur la figure \ref{tps} le workflow de saisie des temps. En fin de mois, chaque consultant doit soumettre un rapport d'activité à son Manager dans lequel il indique le détail de ses journées de travail du mois. Ce comptre rendu d'activité, ou CRA, doit être validé par le Business Manager, puis par le responsable administratif. À chaque étape du processus, l'employé est averti en cas de refus et doit modifier et soumettre à nouveau son CRA jusqu'à validation.
\begin{figure}
	\includegraphics[scale=0.5]{Diagrammes/SaisieDesTempsWF.pdf}
	\caption{Workflow de saisie des temps}
	\label{tps}
\end{figure}


\section{Gestion des relances}
\paragraph{} On peut voir sur la figure \ref{relances} le processus de relance des consultants. Lorsque la fin du mois approche, chaque consultant reçoit un mail lui rappelant de soumettre son Compte Rendu d'activité pour le mois écoulé. À la suite de ce mail, une relance quotidienne est envoyée jusqu'à ce que le consultant ait soumis un CRA correctement complété. Si un certain temps s'écoule sans que le consultant ait soumis son CRA, son bilan annuel en sera impacté.
\begin{figure}
	\includegraphics[scale=0.5]{Diagrammes/Relances.pdf}
	\caption{Processus de relance des consultants sur les Rapports d'Activité}
	\label{relances}
\end{figure}





\chapter{Gestion de la rémunération}
\section{Demande de prime}
\paragraph{} La figure \ref{primes} montre le workflow de demande de primes. Une demande de prime est effectuée par un Manager, pour un consultant. Dans ce cas, on entend par "Manager" le Directeur des Opérations, un Business Manager, Directeur de BU, ou bien un membre de la Direction.
\paragraph{} Dans le cas d'un Business Manager, sa demande passe d'abord par son Directeur de BU, avant d'être validée par la Direction Générale. Sinon, cela passe directement à la Direction Générale pour validation.



\begin{figure}[H]
% \begin{sideways}
	\centering
	\includegraphics[scale=0.5]{Diagrammes/PrimesWF.pdf}
% \end{sideways}
	\caption{Demande et validation de primes} 
	\label{primes}
\end{figure}



\chapter{Gestion des Notes de Frais}
\section{Gestion des notes de frais internes}

\paragraph{} Lorsqu'un interne fait une demande de remboursement de note de frais, elle doit être validée d'abord par son N+1 avant de passer par la Direction générale qui donnera un accord définitif. Le responsable administratif pourra ensuite effectuer le remboursement. Ce workflow est décrit sur la figure \ref{ndfI}.


\begin{figure}[H]
% \begin{sideways}
	\centering
	\includegraphics[scale=0.5]{Diagrammes/NoteDeFraisWFInterne.pdf}
% \end{sideways}
	\caption{Demande et validation de notes de frais internes} 
	\label{ndfI}
\end{figure}

\section{Gestion des notes de frais consultant}

\paragraph{} Lorsqu'un consultant fait une demande de remboursement de note de frais, elle doit être validée par le Responsable Administratif , puis par son Business Manager, avant de passer par la Direction Générale qui donnera un accord définitif. Le responsable administratif pourra ensuite effectuer le remboursement. Ce workflow est décrit sur la figure \ref{ndfC}.

\begin{figure}[H]
% \begin{sideways}
	\centering
	\includegraphics[scale=0.5]{Diagrammes/NoteDeFraisWFConsultant.pdf}
% \end{sideways}
	\caption{Demande et validation de notes de frais consultants}
	\label{ndfC}
\end{figure}


\chapter{Gestion des Affaires}
\section{Assistance Technique Unitaire (ATU)}

\subsection{Mise en place d'une affaire ATU}

\paragraph{}La figure \ref{mepATU} présente le processus de mise en place d'une affaire ATU. Le processus débute avec l'appel d'offre d'un client, auquel un membre de l'équipe commerciale va répondre en proposant un ou plusieurs consultants.\\
Si le client considère le profil convient à ses besoins, il rencontrera le consultant au cours d'un ou plusieurs entretiens, à l'issue desquels il donnera ou non son accord pour démarrer la mission du consultant. S'en suit une phase de contractualisation, visible sur la figure \ref{contractATU}
\paragraph{} Le contrat est rédigé par le client, puis mis à disposition des équipes commerciales sur une plate-forme d'achat dédiée. Le contrat est téléchargé, imprimé, signé, puis transmis au Responsable Administratif pour envoi postal au client.\\
Si nécessaire, le consultant signe des accords supplémentaires, notamment les règles déontologiques et de confidentialité, etc.

\begin{figure}[H]
% \begin{sideways}
	\centering
	\includegraphics[scale=0.3]{Diagrammes/MiseEnPlaceATU.pdf}
% \end{sideways}
	\caption{Mise en place d'une ATU} 
	\label{mepATU}
\end{figure}
	
	
\begin{figure}[H]
% \begin{sideways}
	\centering
	\includegraphics[scale=0.35]{Diagrammes/ContractualisationATU.pdf}
% \end{sideways}
	\caption{Contractualisation d'une ATU} 
	\label{contractATU}
\end{figure}

\subsection{Clôturer une affaire ATU}

\paragraph{} Une affaire ATU peut se clôturer dans trois cas de figure : le consultant arrive au terme de sa mission, le client décide de mettre fin au contrat, ou bien le consultant démissionne.
\paragraph{} Le contrat se conclut donc simplement, avec un arrêt de la facturation et une classification du contrat.

\begin{figure}[H]
% \begin{sideways}
	\centering
	\includegraphics[scale=0.4]{Diagrammes/ClotureATU.pdf}
% \end{sideways}
	\caption{Clôture d'une ATU} 
	\label{contractATU}
\end{figure}


\subsection{Prolonger une affaire ATU}

\paragraph{} La figure \ref{prolongerATU} présente le processus de prolongation d'une affaire ATU. Le responsable administratif reçoit un avenant au contrat de la part du client, le transmet à la direction qui le signe. Le consultant signe les engagements complémentaires, puis le Responsable Administratif renvoie l'avenant et les pièces jointes au client, au format papier ou électronique.

\begin{figure}[H]
% \begin{sideways}
	\centering
	\includegraphics[scale=0.4]{Diagrammes/RenouvelerATU.pdf}
% \end{sideways}
	\caption{Prolongation d'une ATU} 
	\label{prolongerATU}
\end{figure}

\newpage

\section{Assistance Technique Globale}

\subsection{Mise en place d'une affaire ATG}

 \paragraph{} La figure \ref{miseEnPlaceATG} montre le processus de mise en place d'une ATG. Lorsqu'au moins 5 consultants sont déjà placés chez un même client en ATU, le client fait une demande de massification de contrats, et envoie un cahier des charges auquel l'équipe commerciale devra répondre pour permettre la mise en place du contrat. La demande peut également originer de l'équipe commerciale.
 \paragraph{} Pour ce faire, l'équipe commerciale rédige un PAQ présentant les capacités de l'entreprise à répondre aux exigences du client, en s'appuyant sur une batterie de KPI. Si le client estime les éléments présentés suffisants, l'affaire peut être contractualisée. Ce processus est décrit plus en détail dans la partie \ref{subsection:contratATG}.

\begin{figure}[H]
% \begin{sideways}
	\centering
	\includegraphics[scale=0.4]{Diagrammes/MiseEnPlaceATG.pdf}
% \end{sideways}
	\caption{Mise en place d'une ATG} 
	\label{miseEnPlaceATG}
\end{figure}


\subsection{Contractualisation d'une ATG}
\label{subsection:contratATG}

\paragraph{} La figure \ref{contratATG} montre le processus de contractualisation d'une affaire ATG.
\paragraph{} Le contrat est d'abord envoyé par le client à l'équipe commerciale, puis signé par la Direction, et ensuite renvoyé au client par le Responsable Administratif. 


\begin{figure}[H]
% \begin{sideways}
	\centering
	\includegraphics[scale=0.4]{Diagrammes/ContractualisationATG.pdf}
% \end{sideways}
	\caption{Contractualisation d'une ATG} 
	\label{contratATG}
\end{figure}



\subsection{Assignation de ressources à une ATG}

\paragraph{} Dans le cadre d'une ATG, un contact informel avec le client permet de lancer une recherche de ressource, sans appel d'offre ni contractualisation.
\paragraph{}Après contact avec le client, l'équipe commerciale propose un ou plusieurs consultants pouvant correspondre à son besoin. Comme pour une ATU, le client valide les profils au travers d'entretiens, et donne ou non son feu vert. Une Fiche de Gestion du Changement est alors émise par le client, et téléchargée par l'équipe commerciale depuis la plate-forme d'achat dédiée. Le consultant peut ensuite commencer sa mission.
Ce processus est présenté dans la figure \ref{asgnRcesATG}.

\begin{figure}[H]
% \begin{sideways}
	\centering
	\includegraphics[scale=0.3]{Diagrammes/AssignationRessourcesATG.pdf}
% \end{sideways}
	\caption{Assignation de ressources en ATG} 
	\label{asgnRcesATG}
\end{figure}


\subsection{Retrait d'une ressource en ATG}
\paragraph{} la figure \ref{retraitATG} montre le processus de retrait d'une ressource d'une affaire ATG. Cela peut survenir lorsque la mission du consultant s'achève, sur démission du consultant ou sur interruption de la mission par le client. Une FGC est émise, et le consultant arrête la mission.


\begin{figure}[H]
% \begin{sideways}
	\centering
	\includegraphics[scale=0.5]{Diagrammes/RetraitRessourceATG.pdf}
% \end{sideways}
	\caption{Retrait de ressources en ATG} 
	\label{retraitATG}
\end{figure}


\chapter{Gestion des Formations}
\section{Effectuer une session de formation}
\paragraph{} La figure \ref{effSession} montre le processus "Effectuer une session de formation" . La session doit d'abord être organisée (ce processus est décrit en \ref{sec:Orga}), et sera évaluée, au niveau de l'assimilation individuelle pour chaque participant, et au niveau qualitatif sur la session en elle-même. Ce processus est décrit en section \ref{sec:eval}.

\begin{figure}[H]
	\centering
% \begin{sideways}
	\includegraphics[scale=0.4]{Diagrammes/EffectuerSession.pdf}
% \end{sideways}
	\caption{Session de formation} 
	\label{effSession}
\end{figure}
 	



\section{Organiser une session de formation}
\label{sec:Orga}

\paragraph{}La figure \ref{orgaForm} montre le processus de préparation d'une session de formation. La décision d'organiser une session de formation est prise par la direction des opérations, qui contacte le formateur. Celui-ci pourra ensuite préparer les supports de formation et d'évaluation, et les faire valider par la DO.
\paragraph{} En parallèle, la Direction des Opérations définit une date pour la session, en consultation avec le formateur. Une fois la date arrêtée, la DO peut communiquer sur la session. Commence également le processus de gestion des inscriptions, décrit plus en détail dans la section \ref{sec:inscriptions}. Si suffisamment de participants sont inscrits le jour de la formation, la session peut avoir lieu : la salle peut être réservée et les provisions achetées. Sinon, la direction des opérations doit planifier une nouvelle date et organiser de nouvelles inscriptions.
\paragraph{} En dernier lieu, la DO s'occupe d'imprimer les supports de présentation et la liste des participants.
\begin{figure}[H]
	\centering
\begin{sideways}
	\includegraphics[scale=0.32]{Diagrammes/OrganisationSession.pdf}
\end{sideways}
	\caption{Organisation de session de formation} 
	\label{orgaForm}
\end{figure}




\section{Gérer les inscriptions aux formations}
\label{sec:inscriptions}
\paragraph{} La figure \ref{inscriptionForm} présente le processus de gestion des inscriptions aux formations. Lorsque les inscriptions sont ouvertes, chaque consultant qui s'inscrit est automatiquement ajouté à la liste des participants, jusqu'à ce que le nombre de participants soit atteint. Toute inscription survenant après cela sera ajoutée à la liste d'attente, et pourra participer à la formation si des places se libèrent avant que la formation n'ait lieu.
\paragraph{} Un inscrit peut également se désinscrire, ou se retirer de la liste d'attente. Il devra alors répéter l'opération d'inscription s'il souhaite se réinscrire.
\paragraph{} La veille du jour de la formation, la DO envoie un mail de rappel aux employés présents sur la liste des inscrits. 


\begin{figure}[H]
\centering
\begin{sideways}
	\includegraphics[scale=0.5]{Diagrammes/InscriptionsFormation.pdf}
\end{sideways}
	\caption{Inscription à une session de formation}
	\label{inscriptionForm}
\end{figure}


\section{Évaluer une formation}
\label{sec:eval}
\paragraph{} La figure \ref{evalFormation} présente le processus d'évaluation d'une formation. Après une formation, la DO communique aux participants un questionnaire à remplir pour évaluer la qualité de la formation reçue.
\paragraph{} Les aspects évalués sont les suivants :
\begin{itemize}
	\item Adéquation entre la formation et la communication préalable
	\item Adéquation entre la formation et les objectifs fixés
	\item Adaptation de la durée de la formation par rapport au contenu
	\item Adaptation des supports par rapport au contenu
	\item Qualité de la prestation du formateur
	\item Qualité de la planification
	\item Qualité des fournitures
\end{itemize}

\paragraph{} En dernier lieu sont recueillis les souhaits de formations futures. Les réponses aux questionnaires sont prises en compte dans un but d'amélioration continue.


\begin{figure}[H]
\centering
	\begin{sideways}
		\includegraphics[scale=0.4]{Diagrammes/EvaluerUneFormation.pdf}
	\end{sideways}
	\caption{Évaluation d'une session de formation}
	\label{evalFormation}
\end{figure}


\section{Gérer les souhaits de formation}
\paragraph{} La figure \ref{souhaits} présente le processus de gestion des souhaits de formation. Les souhaits de formation sont recueillis au moment de l'évaluation des formations. Sur souhait de la DO, demande récurrente de consultants, ou bien proposition d'un consultant, la DO recherche un-e formateur-rice sur le sujet si nécessaire, et ajoute la formation au catalogue pour répondre à la demande. 

\begin{figure}[H]
\centering
	% \begin{sideways}
		\includegraphics[scale=0.45]{Diagrammes/GestionDesSouhaits.pdf}
	% \end{sideways}
	\caption{Gestion des souhaits de formation}
	\label{souhaits}
\end{figure}


\chapter{Recrutement}
 \paragraph{} Dans cette partie sont évoqués les viviers candidats. Il en existe différentes sortes, et le type de vivier dans lequel est envoyé un candidat dépend de nombreux facteurs qui ne seront pas traités ici.

 \section{Processus de recrutement}
\label{section:recrutFull}
\paragraph{} La figure \ref{recrutementFull} montre le processus de recrutement dans son ensemble.
\paragraph{} Le point de départ est un besoin client relayé par l'équipe commerciale. Ce besoin peut être précis et ponctuel, mais il s'agit plus généralement d'un ensemble de profils génériques, basés sur des compétences recherchées dans les BU.
\paragraph{} L'équipe recrutement effectue ensuite un sourcing (processus détaillé en partie \ref{section:sourcing}) de candidats en se basant sur ces besoins, les répartit entre recruteurs (processus détaillé dans la partie \ref{section:portefeuille} ) puis leur fait passer une suite d'entretiens. Le workflow d'entretien est détaillé dans la partie \ref{sec:entretiens}. Le candidat peut se qualifier pour la suite du processus ou être renvoyé en vivier.
\paragraph{} Dans le cadre d'un recrutement sur profil, la direction fait une proposition au candidat, et il est engagé s'il accepte.
\begin{figure}
	\centering
	\begin{sideways}
	\includegraphics[scale=0.3]{Diagrammes/RecrutementComplet.pdf}
	\end{sideways}
	\caption{Processus de recrutement}
	\label{recrutementFull}	
\end{figure}

\section{Workflow d'entretiens}
\label{sec:entretiens}

\paragraph{} La figure \ref{entretiens} présente le workflow des entretiens.
 
\paragraph{} Si le premier entretien est concluant, le candidat sera présenté à l'équipe commerciale qui donnera son avis sur le candidat, et peut décider de le rencontrer ou non. L'issue de l'entretien déterminera si le candidat recevra une proposition, sur profil ou sur projet, ou s'il sera renvoyé en vivier.

\begin{figure}
	\centering
	% \begin{sideways}
	\includegraphics[scale=0.4]{Diagrammes/EntretiensWF.pdf}
	% \end{sideways}
	\caption{Workflow d'entretiens}
	\label{entretiens}	
\end{figure}

\section{Contacter et attribuer les candidats}
\label{section:portefeuille}
\paragraph{} La figure \ref{gestionCandidats} montre le processus de contact et de répartition des candidats entre les recruteurs.
\paragraph{} Lorsqu'un CV entre dans le scope d'Ingéniance, tous les recruteurs contactent le candidat. Le premier réussissant à obtenir un contact direct l'ajoute à son portefeuille candidat. Il le rencontre ensuite si le candidat est exploitable dans l'immédiat. 
\begin{figure}
	\centering
	% \begin{sideways}
	\includegraphics[scale=0.6]{Diagrammes/GererUnCandidat.pdf}
	% \end{sideways}
	\caption{Gestion des candidats}
	\label{gestionCandidats}	
\end{figure}

\section{Sourcer des candidats}
\label{section:sourcing}
\paragraph{} La figure \ref{sourcing} montre le processus de sourcing de candidats. Les recruteurs cherchent des profils sur les JobBoards et les réseaux sociaux, puis les incorporent dans la base de candidats, en même temps que les profils provenant de candidatures diverses (cooptation, réponse à annonce, candidature spontanée). Si des profils sont déjà présents dans la base, ils seront mis à jour avec les informations les plus récentes. Si ce sont de nouveaux profils, ils seront ajoutés au vivier "candidats à contacter".

\begin{figure}
	\centering
	% \begin{sideways}
	\includegraphics[scale=0.48]{Diagrammes/SourcerUnCandidat.pdf}
	% \end{sideways}
	\caption{Sourcing de candidats}
	\label{sourcing}	
\end{figure}

\section{Publier des offres}

\paragraph{} La figure \ref{publierOffres} montre le processus de publication d'offres d'emploi. Un recruteur rédige l'annonce, qui doit être validée par le directeur du recrutement. Une fois validée, elle est diffusée sur diverses plate-formes d'emploi et sites d'école.


\begin{figure}
	\centering
	% \begin{sideways}
	\includegraphics[scale=0.6]{Diagrammes/PublierOffres.pdf}
	% \end{sideways}
	\caption{Publication d'offres}
	\label{publierOffres}	
\end{figure}




% \section{Assistance Technique Unitaire (ATU)}

\subsection{Mise en place d'une affaire ATU}

\paragraph{}La figure \ref{mepATU} présente le processus de mise en place d'une affaire ATU. Le processus débute avec l'appel d'offre d'un client, auquel un membre de l'équipe commerciale va répondre en proposant un ou plusieurs consultants.\\
Si le client considère le profil convient à ses besoins, il rencontrera le consultant au cours d'un ou plusieurs entretiens, à l'issue desquels il donnera ou non son accord pour démarrer la mission du consultant. S'en suit une phase de contractualisation, visible sur la figure \ref{contractATU}
\paragraph{} Le contrat est rédigé par le client, puis mis à disposition des équipes commerciales sur une plate-forme d'achat dédiée. Le contrat est téléchargé, imprimé, signé, puis transmis au Responsable Administratif pour envoi postal au client.\\
Si nécessaire, le consultant signe des accords supplémentaires, notamment les règles déontologiques et de confidentialité, etc.

\begin{figure}[H]
% \begin{sideways}
	\centering
	\includegraphics[scale=0.3]{Diagrammes/MiseEnPlaceATU.pdf}
% \end{sideways}
	\caption{Mise en place d'une ATU} 
	\label{mepATU}
\end{figure}
	
	
\begin{figure}[H]
% \begin{sideways}
	\centering
	\includegraphics[scale=0.35]{Diagrammes/ContractualisationATU.pdf}
% \end{sideways}
	\caption{Contractualisation d'une ATU} 
	\label{contractATU}
\end{figure}

\subsection{Clôturer une affaire ATU}

\paragraph{} Une affaire ATU peut se clôturer dans trois cas de figure : le consultant arrive au terme de sa mission, le client décide de mettre fin au contrat, ou bien le consultant démissionne.
\paragraph{} Le contrat se conclut donc simplement, avec un arrêt de la facturation et une classification du contrat.

\begin{figure}[H]
% \begin{sideways}
	\centering
	\includegraphics[scale=0.4]{Diagrammes/ClotureATU.pdf}
% \end{sideways}
	\caption{Clôture d'une ATU} 
	\label{contractATU}
\end{figure}


\subsection{Prolonger une affaire ATU}

\paragraph{} La figure \ref{prolongerATU} présente le processus de prolongation d'une affaire ATU. Le responsable administratif reçoit un avenant au contrat de la part du client, le transmet à la direction qui le signe. Le consultant signe les engagements complémentaires, puis le Responsable Administratif renvoie l'avenant et les pièces jointes au client, au format papier ou électronique.

\begin{figure}[H]
% \begin{sideways}
	\centering
	\includegraphics[scale=0.4]{Diagrammes/RenouvelerATU.pdf}
% \end{sideways}
	\caption{Prolongation d'une ATU} 
	\label{prolongerATU}
\end{figure}

\newpage

\section{Assistance Technique Globale}

\subsection{Mise en place d'une affaire ATG}

 \paragraph{} La figure \ref{miseEnPlaceATG} montre le processus de mise en place d'une ATG. Lorsqu'au moins 5 consultants sont déjà placés chez un même client en ATU, le client fait une demande de massification de contrats, et envoie un cahier des charges auquel l'équipe commerciale devra répondre pour permettre la mise en place du contrat. La demande peut également originer de l'équipe commerciale.
 \paragraph{} Pour ce faire, l'équipe commerciale rédige un PAQ présentant les capacités de l'entreprise à répondre aux exigences du client, en s'appuyant sur une batterie de KPI. Si le client estime les éléments présentés suffisants, l'affaire peut être contractualisée. Ce processus est décrit plus en détail dans la partie \ref{subsection:contratATG}.

\begin{figure}[H]
% \begin{sideways}
	\centering
	\includegraphics[scale=0.4]{Diagrammes/MiseEnPlaceATG.pdf}
% \end{sideways}
	\caption{Mise en place d'une ATG} 
	\label{miseEnPlaceATG}
\end{figure}


\subsection{Contractualisation d'une ATG}
\label{subsection:contratATG}

\paragraph{} La figure \ref{contratATG} montre le processus de contractualisation d'une affaire ATG.
\paragraph{} Le contrat est d'abord envoyé par le client à l'équipe commerciale, puis signé par la Direction, et ensuite renvoyé au client par le Responsable Administratif. 


\begin{figure}[H]
% \begin{sideways}
	\centering
	\includegraphics[scale=0.4]{Diagrammes/ContractualisationATG.pdf}
% \end{sideways}
	\caption{Contractualisation d'une ATG} 
	\label{contratATG}
\end{figure}



\subsection{Assignation de ressources à une ATG}

\paragraph{} Dans le cadre d'une ATG, un contact informel avec le client permet de lancer une recherche de ressource, sans appel d'offre ni contractualisation.
\paragraph{}Après contact avec le client, l'équipe commerciale propose un ou plusieurs consultants pouvant correspondre à son besoin. Comme pour une ATU, le client valide les profils au travers d'entretiens, et donne ou non son feu vert. Une Fiche de Gestion du Changement est alors émise par le client, et téléchargée par l'équipe commerciale depuis la plate-forme d'achat dédiée. Le consultant peut ensuite commencer sa mission.
Ce processus est présenté dans la figure \ref{asgnRcesATG}.

\begin{figure}[H]
% \begin{sideways}
	\centering
	\includegraphics[scale=0.3]{Diagrammes/AssignationRessourcesATG.pdf}
% \end{sideways}
	\caption{Assignation de ressources en ATG} 
	\label{asgnRcesATG}
\end{figure}


\subsection{Retrait d'une ressource en ATG}
\paragraph{} la figure \ref{retraitATG} montre le processus de retrait d'une ressource d'une affaire ATG. Cela peut survenir lorsque la mission du consultant s'achève, sur démission du consultant ou sur interruption de la mission par le client. Une FGC est émise, et le consultant arrête la mission.


\begin{figure}[H]
% \begin{sideways}
	\centering
	\includegraphics[scale=0.5]{Diagrammes/RetraitRessourceATG.pdf}
% \end{sideways}
	\caption{Retrait de ressources en ATG} 
	\label{retraitATG}
\end{figure}



\end{document}
