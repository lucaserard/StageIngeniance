\section{Gestion des notes de frais internes}

\paragraph{} Lorsqu'un interne fait une demande de remboursement de note de frais, elle doit être validée d'abord par son N+1 avant de passer à la Direction générale qui donnera un accord définitif. Le responsable administratif pourra ensuite effectuer le remboursement. Ce \textit{workflow} est décrit sur la figure \ref{ndfI}.
\paragraph{} Si le N+1 s'avère être la direction, une seule validation est requise.


\begin{figure}[H]
% \begin{sideways}
	\centering
	\includegraphics[scale=0.5]{Diagrammes/NoteDeFraisWFInterne.pdf}
% \end{sideways}
	\caption{Demande et validation de notes de frais internes} 
	\label{ndfI}
\end{figure}

\section{Gestion des notes de frais consultant}

\paragraph{} Lorsqu'un consultant fait une demande de remboursement de note de frais, elle doit être validée par le Responsable Administratif , puis par son Business Manager, avant de passer par la Direction Générale qui donnera un accord définitif. Le Responsable Financier pourra ensuite effectuer le remboursement. Ce \textit{workflow} est décrit sur la figure \ref{ndfC}.

\begin{figure}[H]
% \begin{sideways}
	\centering
	\includegraphics[scale=0.5]{Diagrammes/NoteDeFraisWFConsultant.pdf}
% \end{sideways}
	\caption{Demande et validation de notes de frais consultants}
	\label{ndfC}
\end{figure}
