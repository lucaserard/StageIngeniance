\section{Répartition des coûts d'IC entre les Business Manager}

\paragraph{} Lorsqu'un consultant entre chez Ingéniance en IC, plusieurs Business Manager peuvent se déclarer interessés par son profil. Il leur revient alors de supporter le coût de son IC.
\paragraph{} Une fois par mois, le responsable administratif divisera donc les coûts des IC de chaque consultant, et les répartira entre les différents Business Manager concernés.
\paragraph{} Ce processus est montré sur la figure \ref{coutsIC}

\begin{figure}
	\centering
	% \begin{sideways}
	\includegraphics[scale=0.6]{Diagrammes/RepartitionCoutsIC.pdf}
	% \end{sideways}
	\caption{Gestion des droits sur les applications}
	\label{coutsIC}
\end{figure}



\section{Création d'un nouveau client}

\paragraph{} Lors d'une ouverture de compte, le BM responsable de l'ouverture créée une fiche client dans le système et saisit les informations minimum dont il dispose (Nom de l'entité, code de l'entité groupe si pertinent, contact client).

\paragraph{} Une fois que le RA reçoit les informations de facturation de la part du client, elle peut compléter les données et créer un compte comptable associé.
\paragraph{} C'est enfin au tour du RF de créer un compte auxiliaire associé au client, et de compléter dans le système les données comptables liées.


\paragraph{}Ce processus est montré sur la figure \ref{nouveauClient}

\begin{figure}
	\centering
	% \begin{sideways}
	\includegraphics[scale=0.6]{Diagrammes/nouveauClient.pdf}
	% \end{sideways}
	\caption{Création d'un nouveau client}
	\label{nouveauClient}
\end{figure}




\section{Création d'un nouveau Fournisseur}

\paragraph{} Lorsqu'un BM reçoit un go client sur un nouvel indépendant, 

\paragraph{} Une fois que le RF reçoit les informations de paiement de la part du fournisseur, il peut créer le compte comptable associé et renseigner ces informations dans ses outils de comptabilité.


\paragraph{}Ce processus est montré sur la figure \ref{nouveauFournisseur}.

\begin{figure}
	\centering
	% \begin{sideways}
	\includegraphics[scale=0.6]{Diagrammes/nouveauFournisseur.pdf}
	% \end{sideways}
	\caption{Création d'un nouveau Fournisseur}
	\label{nouveauFournisseur}
\end{figure}
