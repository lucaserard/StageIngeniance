\section{Demande de prime}
\paragraph{} La figure \ref{primes} montre le workflow de demande de primes. Une demande de prime est effectuée par un Manager, pour un consultant. Dans ce cas, on entend par "Manager" le Directeur des Opérations, un Business Manager, Directeur de BU, ou bien un membre de la Direction.
\paragraph{} Dans le cas d'un Business Manager, sa demande passe d'abord par son Directeur de BU, avant d'être validée par la Direction Générale. Sinon, cela passe directement à la Direction Générale pour validation.



\begin{figure}[H]
% \begin{sideways}
	\centering
	\includegraphics[scale=0.5]{Diagrammes/PrimesWF.pdf}
% \end{sideways}
	\caption{Demande et validation de primes} 
	\label{primes}
\end{figure}

\section{Établissement de la paie}

\paragraph{} La figure \ref{paie} présente le processus d'établissement de la paie. La dernière semaine du mois, le Responsable Financier vérifie les changements éventuels dans les données RH des employés, et met à jour les données de Sage Paie le cas échéant.
\paragraph{} Il peut ensuite importer les données de variables de paie dans Sage Paie. Les données en question comptent : \begin{itemize}
	\item les congés \begin{itemize}
		\item Congés Payés;
		\item Congés Sans Solde;
		\item Arrêt Maladie;
		\item Congé exceptionnel;
		\item RTT;
	\end{itemize}
	\item les paniers repas;
	\item les remboursements de titres de transport;
	\item les primes;
\end{itemize}
\paragraph{} Si des données manquent à l'import (non remise du rapport d'activité mensuel par un consultant), on utilise les données d'absence disponibles grâce aux demandes de congés. Le consultant est averti de l'absence de son RMA, et le responsable Financier prévoit une régulation pour le mois suivant dans le cas où le temps de travail réel du consultant ne correspondrait pas aux données déduites.

\paragraph{} La paie est ensuite calculée à partir de ces variables. Dans le même temps, les compteurs de congés et de RTT sont mis à jour. La paie est finalement virée sur les compte en banque des employés, et le processus se finit avec l'émission des bulletins de paie.

\begin{figure}[H]
\centering
\begin{sideways}
	\includegraphics[scale=0.42]{Diagrammes/Paie.pdf}
\end{sideways}
	\caption{Établissement de la paie} 
	\label{paie}
\end{figure}


\section{Gestion des titres de transports}

\paragraph{} La figure \ref{transports} montre le processus de gestion des titres de transport. Un employé remet son titre de transport au responsable administratif . Ce dernier vérifie le titre, et en saisit la valeur dans le système si il est valide. Dans le cas contraire, le responsable administratif contacte l'employé qui doit le soumettre de nouveau après avoir résolu le problème.

\begin{figure}[H]
\centering
% \begin{sideways}
	\includegraphics[scale=0.5]{Diagrammes/Transport.pdf}
% \end{sideways}
	\caption{Gestion des titres de transport} 
	\label{transports}
\end{figure}

\section{Gestion des paniers repas}
\paragraph{} La figure \ref{repas} montre le processus de gestion des paniers repas. Le responsable administratif décompte les jours entiers travaillés par chaque employé, et en soustrait les jours où l'employé a soumis une note de frais liée à un repas. Le calcul des paniers repas peut ensuite être effectué, sur la base de 5 euros par jour décompté précédemment.



\begin{figure}[H]
\centering
% \begin{sideways}
	\includegraphics[scale=0.5]{Diagrammes/PaniersRepas.pdf}
% \end{sideways}
	\caption{Gestion des paniers repas}
	\label{repas}
\end{figure}






