\documentclass[12pt,twoside]{scrreprt}
\usepackage[T1]{fontenc}
\usepackage[utf8]{inputenc}
\usepackage{lmodern}
\usepackage{textcomp}
\usepackage[francais]{babel, varioref}
\usepackage{graphicx}
\usepackage{listings}
\usepackage{xspace}
\usepackage{amsmath}
\usepackage{amssymb}
\usepackage{calc}
\usepackage{listingsutf8}
\usepackage{color}
\usepackage{xcolor}
\usepackage{afterpage}
% \usepackage[style=verbose-note,backend=bibtex]{biblatex}
\usepackage{url}
\usepackage[top=2.1cm,bottom=2.2cm,left=2cm,right=2cm]{geometry}
\usepackage[final]{pdfpages}
\usepackage{pslatex}
\usepackage{rotating}



% Pour sommaire cliquable 
\usepackage{hyperref} % Créer des liens et des signets
\hypersetup{
colorlinks=true, %colorise les liens
breaklinks=true, %permet le retour à la ligne dans les liens trop longs
urlcolor= blue, %couleur des hyperliens
linkcolor= black, %couleur des liens internes
citecolor=black,  %couleur des références
}

% Fichier de bibliographie
% \bibliography{parties/biblio}

\usepackage{templateIngeniance}

% Titre centre
\renewcommand\titre{Logiciels de recrutement}
\renewcommand\soustitre{Bilan des présentations}

% Auteurs
\date{Avril 2016}

\author{Direction des opérations}

\begin{document}
\initIngeniance


\titleIngeniance{0}{}{0}{150}{40pt}{35pt}
% Remerciements
% \input{parties/00-Contacts.tex}
% \addcontentsline{toc}{chapter}{Contacts}

% Sommaire
\tableofcontents

\chapter*{Informations complémentaires}
\paragraph{Reconnaissance image} Solution de CRM nécessaire pour la plupart.
\paragraph{Espaces Candidat} Attention pour la CNIL : Espace candidat permet l'accès aux données demandé par la loi informatique et liberté 
\paragraph{Réseaux sociaux} Prévoir compte premium pour l'entreprise

% Parties
\chapter{Relation RH}


\paragraph{Intervenant}
\begin{itemize}
	\item Nom : Fatima Ghemougui
	\item Email : fghemougui@horizontalsoftware.com
	\item Téléphone fixe : 0153043236
	\item Téléphone Portable : 0619950820  
\end{itemize}

\section{Offre de service}
\paragraph{} Offre Saas ou on premises disponible. Offre Saas hébergée chez des entreprises externes (ex OVH)
\paragraph{} Solution Web avec fonctionnement sous navigateur.

\section{Gestion des offres}
\paragraph{} Possibilité de poster les offres sur un site de recrutement intégré à notre site actuel, et adapté à notre charte graphique, ou bien en site externe. Front office candidat responsive.
\paragraph{} Possibilité de créer un espace candidat pour permettre l’accès aux données (optionnel mais bien pour la CNIL).
\paragraph{} Analyse sémantique Linkedin, Viadeo, Fichier pour remplir le formulaire côté candidat.
\paragraph{} Les candidatures alimentent le vivier de CV.
\paragraph{} Création d'annonce : champs à renseigner variés. Publication sur les job boards possible, mais utiliser Multiposting (entreprise) pour grande échelle (cher). Publication sur site. Possibilité d'un système d'alerte pour les candidatures sur les annonces urgentes.
\paragraph{} Possibilité d'un WF de validation avant publication d'une annonce.


\section{Gestion des candidatures}

\paragraph{} Système d'infobulle via survol avec les infos intéressantes : paramétrable.
\paragraph{} Fiche candidat affiche infos administratives, compétences et expériences + possibilité d'afficher le CV au format texte brut. Développement possible pour l'affichage CV en format original en parallèle sur la fiche candidat. 
\paragraph{} Possibilité d'exporter un profil vers PDF (Problème d'encodage pendant la démo)
\paragraph{} Workflow de gestion du candidat : Pas de gestion des statuts, uniquement des actions. Historique des actions disponibles. Dernière action affichable et filtrable dans le référentiel candidat et les résultats de recherche.
\paragraph{} Écran d'affichage des "derniers ajouts" permet d'avoir accès aux derniers candidats importés.
\paragraph{} Possibilité d'assigner un candidat à un recruteur via un champ et du filtrage sur les recherches (=non natif)



\section{Communication}
\paragraph{} Possibilité d'exporter les fiches au format PDF.
\paragraph{} Possibilité de transmettre un candidat à un acteur non utilisateur de l'application et d'obtenir son retour sur le profil.
\paragraph{} Possibilité de créer un CV Ingéniance à partir de la fiche candidat.


\section{Recherche et matching}
\paragraph{} Recherche réseaux sociaux, mais ouverture de navigateur avec bing search. Résultats pas super pertinents au premier abord, + manque d'ergonomie (affichage moteur de recherche web).
\paragraph{} Matching avec les CV de la base au moment de la publication de l'annonce.



\section{Import et intégration}
\paragraph{} Import CV et création de fiche candidat : en Drag and drop. Import par une boite mail dispo. Logiciel TextKernel pour l'analyse sémantique (= LeaCV). 
\paragraph{} Import SN géré par la page de recherche via bouton à coté du résultat.


\section{Autres fonctionnalités} 
\paragraph{Contrats de travail :} Développement prévu pour cet été.
\paragraph{Fiches d'entretien}  Customizables pour les recruteurs.
\paragraph{Statistiques} Reporting fait soit par Fiche de Poste, soit sur la BD entière avec requête personnalisable.
\paragraph{Import Monster} Nécessité de passer par des solutions type Multiposting.
\paragraph{Panier Candidat} Possibilité de se créer un panier candidat pour shortlister certains profils.

\section{Projet}
\paragraph{} Paramétrage sur environ 6 semaines.
\paragraph{} Reprise des données prise en charge par Horizontal Software.
\paragraph{} Volume de données ne pose pas de problème.


% \section{Conclusion sur la solution}
% \paragraph{} Solution couvre la plupart de nos besoins, en revanche gros défaut au niveau du workflow de recrutement. Prendre garde au


\chapter{Talent Profiler}


\paragraph{Intervenant}
\begin{itemize}
	\item Nom : Stéphane Chalençon
	\item Email : stephane.chalencon@koltech.com
	\item Téléphone fixe : 0175772472  
\end{itemize}

\section{Offre de service}
\paragraph{} Full SaaS. Interface Web sur navigateur.
\paragraph{} Actuellement en fin de version 2. La V3 serait disponible en juillet. Changement complet de technologie : passage de J2EE à PHP donc risque d'instabilité


\section{Gestion des offres}

\paragraph{} Intégration des offres au site : ouverture de WS pour récupérer les données, intégration dans le site à notre charge.
\paragraph{} Diffusion sur les job boards bien gérée, mais absence de LesJeudis.

\section{Gestion des candidatures}
\paragraph{} Tableau de bord en page d'accueil, candidatures séparés par statut . Colonnes modulables.
\paragraph{} Workflow actions/statuts bien géré, possibilité d'actions automatiques.
\paragraph{} Possibilité de partage avec les recruteurs avec système d'alerte.
\paragraph{} Distinction entre candidat et candidature : un candidat peut postuler plusieurs fois sans qu'une fiche candidat soit recréée, sauvegarde des différentes candidatures.
\paragraph{} Affichage du CV sur le coté de la fiche candidat, au format source. CV images affichés même sans parsing.
\paragraph{} Possibilité d'envoyer des candidatures à un non utilisateur, via un lien personnel donnant accès à un espace privé avec candidatures triés par poste/profil et possibilité de laisser des commentaires en plus de l'avis.


\section{Communication}
\paragraph{} Génération de CV et de documents pas disponible
\paragraph{} Modèles d'email disponibles, mais pas sûr pour le publipostage.
\paragraph{} Modèle SMS également, mais achat de packs SMS nécessaire. Programmation d'envoi SMS pas certaine.
\paragraph{} Intégration avec outlook (mail + calendrier)
\paragraph{} Communication de masse sur sélection de candidats dispo.
\paragraph{} Partenariat d'envoi email transparent et compris dans le prix de vente (pas de limitation de volume).


\section{Recherche et matching}
\paragraph{} Pas de recherche sur les réseaux sociaux
\paragraph{} Dictionnaire de synonymes sur les mots clés (paramétrable).


\section{Intégration et import}
\paragraph{} Import Linkedin et Viadeo fait sur navigateur, via plug-in : bien géré. 

\section{Autres fonctionnalités}

\paragraph{Reporting et statistiques :} Disponible mais moins personnalisable que Relation RH.




\chapter{AD MEN}

\paragraph{Intervenant}
\begin{itemize}
	\item Nom : François Housseau
	\item Email : fhousseau@ad-rh.com
	\item Téléphone fixe :
\end{itemize}

%---------------------------------------------------------------------------------------------------------------------------------------------

\section{Offre de service}
\paragraph{} On premises uniquement, outil windows.

\paragraph{} Licence par utilisateur plutôt que par poste.
\paragraph{} Application mobile et tablette  : ADMen Air
\paragraph{Ouverture API} Données hébergées sur nos serveurs = BD requêtable directement. Possibilité de mise en place de vues SQL, de flux XML ou RSS (coûts supplémentaires)

\section{Gestion des offres}
\paragraph{} Création de fiches de poste basique
\paragraph{} Intégration sur site Web à notre charge
\paragraph{} AdMen Web Application intégrable au site.
\paragraph{} Partenariat Multiposting, Broadbin et Talentplug (mais abonnements à notre charge)




\section{Gestion des candidatures} 
\paragraph{} Gestion par mission : liste de candidats, documents et annonces associés.
\paragraph{} Workflow de recrutement très bien géré, actions et statuts personnalisables.
\paragraph{} Gestion des profils : identification des doublons de candidatures
\paragraph{} Affichage de la fiche candidat sans CV en parallèle.
\paragraph{} Complétion des profils avec conflit de données très bien gérée.
\paragraph{} Possibilité de créer des \textit{triggers} avec action à la clé.

\section{Communication}
\paragraph{} Envoi de candidatures à des commerciaux bien géré, soit par lien chiffré, soit sur leur compte. Possibilité de joindre CR d'entretien.
\paragraph{Modèles de documents} Génération de documents avec publipostage possible sur un grand nombre de documents types. Entre autre CV et contrat de travail.
\paragraph{} Mailing de masse disponible, avec modèles de mail et publipostage.


\section{Import et intégration}
\paragraph{} Parsing complet, manque les fichiers images.
\paragraph{} Import FTP et mail disponibles.
\paragraph{} Intégration Linkedin plus disponible, mais Viadeo bien gérée, avec recherche intégrée.


\section{Autres fonctionnalités}
\paragraph{Gestion des compétences} Organisation en arborescence : jusqu'à 4 niveaux de profondeur, ce qui permet de mettre en place des niveaux. Attention cependant, manque notion d'ordre entre les niveaux.
\paragraph{Post it :} Possibilité de coller des post-it avec du texte sur tous les objets de l'application.
\paragraph{Fonctionnalités CRM, chasse etc :} Système de création de société, liée à Corporama et , de rechercher tous les profils liés à une société sur plusieurs critères.
\paragraph{Intégration partenaires :} \begin{itemize}
	\item AssessFirst : Personnalité
	\item EasyRecrue : questions vidéos
	\item Insee : salaire cadres [à venir]
	\item eTesting, Isograd : tests techniques
	\item Corporama : infos sociétés
	\item VerifDiploma
\end{itemize}

\paragraph{Modules supplémentaires :} Gestion des sociétés, des documents, de la facturation, de saisie des temps.

\paragraph{Plugin MS} Intégration complète dans la suite MS : boutons sur tous les logiciels pour import/export etc, synchronisation outlook et plug-in IE (seul navigateur compatible).



\chapter{Bilan provisoire}



\section{Solution RH}

\subsection{Points forts}
\begin{itemize}
	\item Système de survol avec affichage de texte : ergonomie
	\item Très bon sur la gestion des offres d'emploi avec intégration sur le site Web
	\item Écran d'affichage des derniers profils
	\item Fiches d'entretien
	\item Reprise des données prise en charge
	\item Full SaaS, solution Web
\end{itemize}
\subsection{Points faibles}
\begin{itemize}
	\item Workflow géré avec des actions uniquement
\end{itemize}


\section{TalentPofiler}

\subsection{Points forts}
\begin{itemize}
	\item Full SaaS, solution Web
	\item Affichage de la fiche candidat avec CV en parallèle
	\item Look and Feel excellent sur V3
	\item Workflow avec actions bien géré
	\item Espace privé pour les commerciaux
\end{itemize}
\subsection{Points faibles}
\begin{itemize}
	\item Changement complet de techno entre V2 et V3 = risque d'instabilité
	\item Publipostage douteux
	\item Pas de génération de CV ou de docs
\end{itemize}


\section{AdMen}

\subsection{Points forts}
\begin{itemize}
	\item Présence de triggers
	\item Référentiel de compétences
	\item Recherche Viadeo intégrée
\end{itemize}
\subsection{Points faibles}
\begin{itemize}
	\item On premises uniquement
	\item Uniquement IE
	\item Outil peu adapté à notre mode de fonctionnement
\end{itemize}

% Bibliographie
%\bibliographystyle{plain}
% \printbibliography

% Annexes
%\clearpage
% \appendix
% \chapter{Annexes}
% \input{parties/10-Annexe.tex}

\end{document}
